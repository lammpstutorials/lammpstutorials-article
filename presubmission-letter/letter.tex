\documentclass{article}

\usepackage[T1]{fontenc}
\usepackage[utf8]{inputenc}
\usepackage{lmodern}
\usepackage{verbatim}
\usepackage{graphicx}
\usepackage{amsmath}
\usepackage{amssymb}
\usepackage{amsthm}
\usepackage{tabularx}
\usepackage{multirow}
\usepackage{multicol}
\usepackage{fancyvrb}
\usepackage{float}
\usepackage{listings}
\usepackage{xcolor}
\usepackage{array}
\usepackage{booktabs}
\usepackage{times}
\usepackage{subcaption}
\usepackage{mathtools}
\usepackage{microtype}
\usepackage{tcolorbox}
\usepackage{newverbs}
\usepackage[
            colorlinks = true,
            linkcolor = blue,
            urlcolor  = blue,
            citecolor = violet,
            anchorcolor = blue]{hyperref}

\usepackage[
	a4paper, % Paper size
	top=1in, % Top margin
	bottom=1in, % Bottom margin
	left=1in, % Left margin
	right=1in, % Right margin
]{geometry}

\setlength{\parindent}{0pt} % Paragraph indentation
\setlength{\parskip}{1em} % Vertical space between paragraphs

\usepackage{graphicx} % Required for including images

\usepackage{fancyhdr} % Required for customizing headers and footers

\fancypagestyle{firstpage}{%
	\fancyhf{} % Clear default headers/footers
	\renewcommand{\headrulewidth}{0pt} % No header rule
	\renewcommand{\footrulewidth}{1pt} % Footer rule thickness
}

\fancypagestyle{subsequentpages}{%
	\fancyhf{} % Clear default headers/footers
	\renewcommand{\headrulewidth}{1pt} % Header rule thickness
	\renewcommand{\footrulewidth}{1pt} % Footer rule thickness
}

\AtBeginDocument{\thispagestyle{firstpage}} % Use the first page headers/footers style on the first page
\pagestyle{subsequentpages} % Use the subsequent pages headers/footers style on subsequent pages

%----------------------------------------------------------------------------------------

\begin{document}

%----------------------------------------------------------------------------------------
%	FIRST PAGE HEADER
%----------------------------------------------------------------------------------------

\includegraphics[height=0.2\textwidth]{liphy.jpg} % Logo
\hspace{1.55cm}
\includegraphics[height=0.2\textwidth]{uga.png} % Logo
\hspace{1.55cm}
\includegraphics[height=0.2\textwidth]{cnrs.png} % Logo

\vspace{-1em} % Pull the rule closer to the logo

\rule{\linewidth}{1pt} % Horizontal rule

\bigskip\bigskip % Vertical whitespace

%----------------------------------------------------------------------------------------
%	YOUR NAME AND CONTACT INFORMATION
%----------------------------------------------------------------------------------------

\hfill
\begin{tabular}{l @{}}
	\today \bigskip\\ % Date
	Simon Gravelle \\
    Laboratoire LIPhy (UGA \& CNRS) \\
	140 Rue de la Physique \\
    38402 Saint-Martin-d'Hères \\
    France \\
	Email: simon.gravelle@cnrs.fr
\end{tabular}

\bigskip % Vertical whitespace

%----------------------------------------------------------------------------------------
%	ADDRESSEE AND GREETING
%----------------------------------------------------------------------------------------

\begin{tabular}{@{} l}
	Prof. Mala Radhakrishnan \\
    Wellesley College \\
    106 Central St. Wellesley \\
    MA 02481\\
    United States of America \\
    Email: mradhakr@wellesley.edu
\end{tabular}

\bigskip 

Subject: pre-submission inquiry for \textit{A Set of Tutorials for the LAMMPS Simulation Package}

\bigskip 

Dear Editor,

\bigskip 

Over the last three years, I have developed an open-source website, \href{https://lammpstutorials.github.io/}{lammpstutorials.github.io}, dedicated  to facilitating the learning of the LAMMPS molecular dynamics code. The website has now achieved a significant level of completion, and I would like to publish its content as a peer-reviewed article. I firmly believe that the \textit{Living Journal of Computational Molecular Science}, and in particularly its \textit{Tutorials and Training Articles} section, is the ideal venue for this work.

The seven tutorials featured on \href{https://lammpstutorials.github.io/}{lammpstutorials.github.io} are crafted to make LAMMPS more accessible to new users. The first four tutorials cover fundamental aspects of running molecular simulations in LAMMPS, including systems such as a simple fluid and a carbon nanotube. The subsequent three tutorials delve into advanced simulations, focusing on the use of a reactive force field, umbrella sampling, and grand canonical Monte Carlo methods.

With this letter, I respectfully request your consideration for the publication of a manuscript, titled \textit{A Set of Tutorials for the LAMMPS Simulation Package}, in the \textit{Living Journal of Computational Molecular Science}. I believe that these tutorials will be of interest to your readership and will support the journal's mission of advancing computational molecular science.

Thank you for considering my submission. I am looking to the opportunity to contribute to your journal.

\bigskip % Vertical whitespace

Sincerely yours,

\bigskip % Vertical whitespace

Simon Gravelle

\end{document}

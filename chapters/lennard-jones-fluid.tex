\subsection{Tutorial 1: Lennard-Jones fluid}
\label{lennard-jones-label}

\noindent The objective of this tutorial is to perform the simulation of a binary fluid using LAMMPS. The system is a Lennard-Jones fluid made of neutral particles with two different diameters in a cubic box with periodic boundary conditions. The temperature of the system is maintained using a Langevin thermostat \cite{schneider1978molecular}, and basic quantities are extracted from the system, including the potential and kinetic energies. 

\subsubsection{My first input}

\noindent To run a simulation using LAMMPS, one needs to write a series of commands in an input script. For clarity, this script will be divided into five categories which we are going to fill up one by one. Create a folder, call it \textit{my-first-input/}, and then create a blank text file in it called \textit{input.lammps}. Copy the following lines in \textit{input.lammps}, where a line starting with a brace ($\#$) is a comment that is ignored by LAMMPS:
{\small  \begin{verbatim}
# PART A - ENERGY MINIMIZATION
# 1) Initialization
# 2) System definition
# 3) Simulation settings
# 4) Visualization
# 5) Run
\end{verbatim} }
\noindent These five categories are not required in every input script, and should not necessarily be in that exact order. For instance, parts 3 and 4 could be inverted, or part 4 could be omitted. Note however that LAMMPS reads input files from top to bottom, therefore the \textit{Initialization} and \textit{System definition} categories must appear at the top of the input, and the \textit{Run} category at the bottom.

\paragraph{System initialization}
In the first section of the script, called \textit{Initialization}, let us indicate to LAMMPS the most basic information about the simulation, such as:
\begin{itemize}
\item the conditions at the boundaries of the box (e.g. periodic or non-periodic),
\item the type of atoms (e.g. uncharged single dots or spheres with angular velocities).
\end{itemize}
Enter the following lines in \textit{input.lammps}:
{\small \begin{verbatim}
# 1) Initialization
units lj
dimension 3
atom_style atomic
pair_style lj/cut 2.5
boundary p p p
\end{verbatim}}
The first line, \textit{units lj}, indicates that we want to use the system of unit called \textit{LJ}, for Lennard-Jones, for which all quantities are unitless. The second line, \textit{dimension 3}, indicates that the simulation is 3D. The third line, \textit{atom$\_$style atomic}, that the \textit{atomic} style
will be used, therefore each atom is just a dot with a mass. The fourth line, \textit{pair$\_$style lj/cut 2.5}, indicates that atoms will be interacting through a Lennard-Jones potential with a cut-off equal to $r_c = 2.5$ (unitless) \cite{wang2020lennard,fischer2023history}:
$$E_{ij} (r) = 4 \epsilon_{ij} \left[ \left( \dfrac{\sigma_{ij}}{r} \right)^{12} - \left( \dfrac{\sigma_{ij}}{r} \right)^{6} \right], ~ \text{for} ~ r < r_c,$$
where $r$ is the inter-particles distance, $\epsilon_{ij}$ the depth of potential well that sets the interaction strength, and $\sigma_{ij}$ the distance parameter, or particle effective size. Here, the indexes \textit{ij} refer to the particle types \textit{i} and \textit{j}. The last line, \textit{boundary p p p}, indicates that the periodic boundary conditions will be used along all three directions of space (the 3 \textit{p} stand for \textit{x}, \textit{y}, and \textit{z}, respectively).

\paragraph{System definition}
Let us fill the \textit{System definition} category of the input script:
{\small \begin{verbatim}
# 2) System definition
region simulation_box block -20 20 -20 20 -20 20
create_box 2 simulation_box
create_atoms 1 random 1500 341341 simulation_box
create_atoms 2 random 100 127569 simulation_box
\end{verbatim}}
\noindent The first line, \textit{region simulation$\_$box (...)}, creates a region named \textit{simulation$\_$box} that is a block (i.e. a rectangular cuboid) that extends from -20 to 20 (no unit) along all 3 directions of space. The second line, \textit{create$\_$box 2 simulation$\_$box}, creates a simulation box based on the region \textit{simulation$\_$box} with \textit{2} types of atoms. The third line, \textit{create$\_$atoms (...)} creates 1500 atoms of type 1 randomly within the region \textit{simulation$\_$box}. The integer \textit{341341} is a seed that can be changed in order to create different
initial conditions for the simulation. The fourth line creates 100 atoms of type 2.

\paragraph{Simulation Settings}
Let us fill the \textit{Simulation Settings} category section of the \textit{input} script:
{\small \begin{verbatim}
# 3) Simulation settings
mass 1 1
mass 2 1
pair_coeff 1 1 1.0 1.0
pair_coeff 2 2 0.5 3.0
\end{verbatim}}
The two first commands, \textit{mass (...)}, attribute a mass equal to 1 (unitless) to both atoms of type 1 and 2. Alternatively, one could have written these two commands into one single line: \textit{mass $\star 1$}, where the star symbol means \textit{all} the atom types of the simulation.  The third line, \textit{pair$\_$coeff 1 1 1.0 1.0}, sets the Lennard-Jones coefficients for the interactions between atoms of type 1, respectively the energy parameter $\epsilon_{11} = 1.0$ and the distance parameter $\sigma_{11} = 1.0$. Similarly, the last line sets the Lennard-Jones coefficients for the interactions between atoms of type 2, $\epsilon_{22} = 0.5$, and $\sigma_{22} = 3.0$. By default, LAMMPS calculates the cross coefficients between the different atom types using geometric average: $\epsilon_{ij} = \sqrt{\epsilon_{ii} \epsilon_{jj}}$, $\sigma_{ij} = \sqrt{\sigma_{ii} \sigma_{jj}}$. 


\paragraph{Energy minimization}
The system is now fully parametrized. Let us fill the two last remaining sections by adding the following lines to \textit{input.lammps}:
\begin{verbatim}
# 4) Visualization
thermo 10
thermo_style custom step temp pe ke etotal press

# 5) Run
minimize 1.0e-4 1.0e-6 1000 10000
\end{verbatim}
The \textit{thermo} command asks LAMMPS to print thermodynamic information (e.g. temperature, energy) in the terminal every given number of steps, here 10 steps. The \textit{thermo$\_$style custom} requires LAMMPS to print the system temperature (\textit{temp}), potential energy (\textit{pe}), kinetic energy (\textit{ke}), total energy (\textit{etotal}), and pressure (\textit{press}). Finally, the \textit{minimize} line asks LAMMPS to perform an energy minimization of the system. By default, LAMMPS uses the conjugate gradient (CG) algorithm \cite{hestenes1952methods}.

Run the simulation by typing in a terminal:
\begin{verbatim}
lmp -in input.lammps
\end{verbatim}
where the command \textit{lmp} is linked to a compiled version of LAMMPS. As the simulation progresses, the potential energy can be seen to decrease from a large positive value to to a negative value. The initially large and positive value of the potential energy was expected because the atoms have been created at random positions within the simulation box and some of them are probably overlapping, resulting in a large initial energy which is the consequence of the repulsive part of the Lennard-Jones interaction potential. As the energy minimization progresses, the energy rapidly decreases and reaches a negative value, indicating that the atoms have been displaced at reasonable distances from each other.


\subsection{Molecular dynamics}
The system is now ready. Let us continue filling up the input script and adding commands to perform a molecular dynamics simulation that will start from the final state of the previous energy minimization step. In the same input script, after the \textit{minimization} command, add the following
lines:
\begin{verbatim}
# PART B - MOLECULAR DYNAMICS
# 4) Visualization
thermo 50
\end{verbatim}
Since LAMMPS reads the input from top to bottom, these lines will be executed after the energy minimization. There is no need to re-initialize or re-define the system. The \textit{thermo} command is called a second time within the same input, so the previously entered value of 10 will be replaced by
the value of 50 as soon as \textit{PART B} starts. Then, let us add a second \textit{Run} section:
\begin{verbatim}
# 5) Run
fix mynve all nve
fix mylgv all langevin 1.0 1.0 0.1 1530917
timestep 0.005
run 10000
\end{verbatim}
The \textit{fix nve} is used to update the positions and the velocities of the atoms in the group \textit{all} at every step. The group \textit{all} is a default group that contains every atom. The second fix applies a Langevin thermostat to the atoms of the group \textit{all}, with a desired initial temperature of 1.0 (unitless), and a final temperature of 1.0 as well \cite{schneider1978molecular}. A \textit{damping} parameter of 0.1 is used. The \textit{damping} parameter determines how rapidly the temperature is relaxed to its desired value. The number \textit{1530917} is a seed, you can change it to perform statistically independent simulations. Finally, the last two lines set the value of the \textit{timestep} and the number of steps for the *run*, respectively, corresponding to a total duration of 50 (unitless).
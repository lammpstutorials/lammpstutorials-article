\chapter{Nanosheared electrolyte}
\label{sheared-confined-label}

\noindent \vspace{-1cm} \noindent \textcolor{graytitle}{\textit{{\Large Aqueous NaCl solution sheared by two walls}}\vspace{0.5cm} }

\vspace{0.25cm} \noindent The objective of this tutorial is to
simulate an electrolyte nanoconfined and sheared by two walls.
Some properties of the sheared fluid, such as its
density and velocity profiles, will be extracted. 

\vspace{0.25cm} \noindent This tutorial illustrates some key aspects of
combining a fluid and a solid in the same simulation.
A major difference with \hyperref[all-atoms-label]{Polymer in water} is that
here a rigid four points water model named TIP4P is used \cite{abascal2005general}.
TIP4P is one of the most common water models due to its high accuracy.

\vspace{0.25cm} \noindent If you are completely new to LAMMPS, I recommend that
you follow this tutorial on a simple \hyperref[lennard-jones-label]{Lennard Jones fluid} first.

\section{System preparation}
\noindent The fluid and walls must first be generated, and then
equilibrated at reasonable temperature and pressure.

\subsection{System generation}
\noindent Create a new folder called \textit{systemcreation/}.
Within \textit{systemcreation/}, open a blank file
called \textit{input.lammps}, and copy the following
lines into it:

\begin{lcverbatim}
units real
atom_style full
bond_style harmonic
angle_style harmonic
pair_style lj/cut/tip4p/long 1 2 1 1 0.1546 12.0
kspace_style pppm/tip4p 1.0e-4
\end{lcverbatim}

\noindent These lines are used to define the most basic parameters,
including the \textit{atom}, \textit{bond}, and \textit{angle} styles, as well as 
interaction potential. Here \textit{lj/cut/tip4p/long} imposes
a Lennard Jones potential with a cut-off at $12\,\text{$\text{\AA{}}$}$
and a long-range Coulomb potential. 

\vspace{0.25cm} \noindent So far, the commands are relatively similar to the 
previous tutorial (\hyperref[all-atoms-label]{Polymer in water}),
with two major differences; the use of \textit{lj/cut/tip4p/long}
and \textit{pppm/tip4p}, instead of \textit{lj/cut/coul/long} and pppm.
These two tip4p-specific commands allow us to 
model a four-point water molecule without explicitly 
defining the fourth massless atom \textit{M}. The value of 
$0.1546\,\text{$\text{\AA{}}$}$ corresponds
to the \textit{O-M} distance and is 
given by the water model. Here, \href{http://www.sklogwiki.org/SklogWiki/index.php/TIP4P/2005_model_of_water}{TIP4P-2005} is used \cite{abascal2005general}.

\begin{tcolorbox}[colback=mylightblue!5!white,colframe=mylightblue!75!black,title=About lj/cut/tip4p/long pair style]

\vspace{0.25cm} \noindent The \textit{lj/cut/tip4p/long} pair style is similar to the conventional 
Lennard Jones + Coulomb interaction, except that it is made specifically 
for four-point water model (tip4p). The atoms of the water model
will be type 1 (O) and 2 (H). All the other atoms of the simulations 
are treated \textit{normally} with long-range Coulomb interaction.
\end{tcolorbox}

\noindent Let us create the box by adding the following lines to \textit{input.lammps}:

\begin{lcverbatim}
lattice fcc 4.04
region box block -3 3 -3 3 -7 7
create_box 5 box &
bond/types 1 &
angle/types 1 &
extra/bond/per/atom 2 &
extra/angle/per/atom 1 &
extra/special/per/atom 2
\end{lcverbatim}

\noindent The \textit{lattice} command defines the unit
cell. Here, the face-centered cubic (fcc) lattice with a scale factor of
4.04 has been chosen for the future positioning of the atoms
of the walls.

\vspace{0.25cm} \noindent The \textit{region} command defines a geometric
region of space. By choosing \textit{xlo=-3} and \textit{xhi=3}, and
because we have previously chosen a lattice with a scale
factor of 4.04, the region box extends from -12.12 Å to 12.12 Å
along the x direction.

\vspace{0.25cm} \noindent The \textit{create$\_$box} command creates a simulation box with 5 types of atoms:
the oxygen and hydrogen of the water molecules,
the two ions ($\text{Na}^+$,
$\text{Cl}^-$), and the
atom of the walls. The \textit{create$\_$box} command extends over 6 lines thanks to the
$\&$ character. The second and third lines are used to
indicate that the simulation contains 1 type of bond and 1
type of angle (both required by the water molecule). The parameters for
these bond and angle constraints will be given later. The
three last lines are for memory allocation.

\vspace{0.25cm} \noindent Now, we can add atoms to the system. First, let us create two
sub-regions corresponding respectively to the two solid
walls, and create a larger region from the union of the two
regions. Then, let us create atoms of type 5 (the wall) within the two
regions. Add the following lines to \textit{input.lammps}:

\begin{lcverbatim}
region rbotwall block -3 3 -3 3 -4 -3
region rtopwall block -3 3 -3 3 3 4
region rwall union 2 rbotwall rtopwall
create_atoms 5 region rwall
\end{lcverbatim}

\noindent Atoms will be placed in the positions of the previously
defined lattice, thus forming fcc solids.

\vspace{0.25cm} \noindent In order to add the water molecules, first
download the \href{https://lammpstutorials.github.io/lammpstutorials-inputs/level2/nanosheared-electrolyte/systemcreation/RigidH2O.txt}{molecule template}
and place it within \textit{systemcreation/}. The template contains all the
necessary information concerning the water molecule, such as
atom positions, bonds, and angles.

\vspace{0.25cm} \noindent Add the following lines to \textit{input.lammps}:

\begin{lcverbatim}
region rliquid block INF INF INF INF -2 2
molecule h2omol RigidH2O.txt
create_atoms 0 region rliquid mol h2omol 482793
\end{lcverbatim}

\noindent Within the last four lines, a \textit{region} named \textit{rliquid} for depositing the
water molecules are created based on the last defined lattice, which is \textit{fcc 4.04}. 

\vspace{0.25cm} \noindent The \textit{molecule} command opens up the molecule template named
\textit{RigidH2O.txt}, and names the associated molecule \textit{h2omol}.

\vspace{0.25cm} \noindent Molecules are created on the \textit{fcc 4.04} lattice
by the \textit{create$\_$atoms} command. The
first parameter is '0', meaning that the atom ids from the
\textit{RigidH2O.txt} file will be used.
The number \textit{482793} is a seed that is
required by LAMMPS, it can be any positive integer.

\vspace{0.25cm} \noindent Finally, let us create 30 ions (15 $\text{Na}^+$
and 15 $\text{Cl}^-$)
in between the water molecules, by adding the following commands to \textit{input.lammps}:

\begin{lcverbatim}
create_atoms 3 random 15 52802 rliquid overlap 0.3 maxtry 500
create_atoms 4 random 15 90182 rliquid overlap 0.3 maxtry 500
set type 3 charge 1
set type 4 charge -1
\end{lcverbatim}

\noindent Each \textit{create$\_$atoms} command will add 15 ions at random positions
within the 'rliquid' region, ensuring that there is no \textit{overlap} with existing
molecules. Feel free to increase or decrease the salt
concentration by changing the number of desired ions. To keep the system charge neutral,
always insert the same number of 
$\text{Na}^+$
and $\text{Cl}^-$,
unless there are other charges in the system.

\vspace{0.25cm} \noindent The charges of the newly added ions are specified by the two \textit{set} commands.

\vspace{0.25cm} \noindent Before starting the simulation, we still need to define the parameters of the simulation: the mass
of the 5 atom types (O, H, $\text{Na}^+$, $\text{Cl}^-$, and wall), the
pairwise interaction parameters (here, the parameters for the
Lennard-Jones potential), and the bond and angle parameters.
Copy the following line into \textit{input.lammps}:

\begin{lcverbatim}
include ../PARM.lammps
include ../GROUP.lammps
\end{lcverbatim}

\noindent Create a new text file, call it \textit{PARM.lammps}, and copy it
next to the \textit{systemcreation/} folder. Copy the following lines
into PARM.lammps:

\begin{lcverbatim}
mass 1 15.9994 # water
mass 2 1.008 # water
mass 3 28.990 # ion
mass 4 35.453 # ion
mass 5 26.9815 # wall

pair_coeff 1 1 0.185199 3.1589 # water
pair_coeff 2 2 0.0 1.0 # water
pair_coeff 3 3 0.04690 2.4299 # ion
pair_coeff 4 4 0.1500 4.04470 # ion
pair_coeff 5 5 11.697 2.574 # wall
pair_coeff 1 5 0.4 2.86645 # water-wall

bond_coeff 1 0 0.9572 # water

angle_coeff 1 0 104.52 # water
\end{lcverbatim}

\noindent Each \textit{mass} command assigns a mass in grams/mole to an atom type. Each
\textit{pair$\_$coeff} assigns respectively the depth of the LJ potential
(in Kcal/mole), and the distance (in Ångstrom) at which the
particle-particle potential energy is 0.

\begin{tcolorbox}[colback=mylightblue!5!white,colframe=mylightblue!75!black,title=About the parameters]

\vspace{0.25cm} \noindent The parameters for water
correspond to the TIP4P/2005 water model, for which only 
the oxygen interacts through Lennard-Jones potential, and the parameters
for $\text{Na}^+$ and $\text{Cl}^-$ are
from the CHARMM-27 force field :cite:`mackerell2000development`.
\end{tcolorbox}

\noindent As already seen in previous tutorials, 
and with the important exception of \textit{pair$\_$coeff 1 5},
only pairwise interaction between atoms of
identical types were assigned. By default, LAMMPS calculates
the pair coefficients for the interactions between atoms
of different types (i and j) by using geometrical
average: $\epsilon_{ij} = (\epsilon_{ii} + \epsilon_{jj})/2$, 
$\sigma_{ij} = (\sigma_{ii} + \sigma_{jj})/2.$
Other rules for cross coefficients can be set with the
\textit{pair$\_$modify} command, but for the sake of simplicity,
the default option is kept here.

\vspace{0.25cm} \noindent By default, the value
of $\epsilon_\text{1-5} = 5.941\,\text{kcal/mol}$ would
be extremely high (compared to the water-water
energy $\epsilon_\text{1-1} = 0.185199\,\text{kcal/mol}$),
which would make the surface extremely hydrophilic.
The walls were made less hydrophilic by reducing the 
LJ energy of interaction $\epsilon_\text{1-5}$.

\vspace{0.25cm} \noindent The \textit{bond$\_$coeff}, which is here used for the O-H bond of the water
molecule, sets both the energy of the harmonic
potential and the equilibrium distance in Ångstrom. The
value is \textit{0} for the energy because we are going to use a
rigid model for the water molecule. The shape of the
molecule will be preserved later by the \textit{shake} algorithm.
Similarly, the angle coefficient here for the H-O-H angle
of the water molecule sets the energy of the harmonic
potential (also 0) and the equilibrium angle is in degree.

\vspace{0.25cm} \noindent Let us also create another file called \textit{GROUP.lammps} next
to \textit{PARM.lammps}, and copy the following lines into it:

\begin{lcverbatim}
group H2O type 1 2
group Na type 3
group Cl type 4
group ions union Na Cl
group fluid union H2O ions

group wall type 5
region rtop block INF INF INF INF 0 INF
region rbot block INF INF INF INF INF 0
group top region rtop
group bot region rbot
group walltop intersect wall top
group wallbot intersect wall bot
\end{lcverbatim}

\noindent To avoid high density and pressure,
let us add the following lines to \textit{input.lammps}
to delete a few of the water molecules:

\begin{lcverbatim}
delete_atoms random fraction 0.15 yes H2O NULL 482793 mol yes
\end{lcverbatim}

\noindent Finally, add the following lines to \textit{input.lammps}:

\begin{lcverbatim}
run 0

write_data system.data
write_dump all atom dump.lammpstrj
\end{lcverbatim}

\noindent With \textit{run 0}, the simulation will run for 0 steps, which is
enough for creating the system and saving the final state.

\vspace{0.25cm} \noindent The \textit{write$\_$data} creates a file named \textit{system.data}
containing all the information required to restart the
simulation from the final configuration generated by this
input file.

\vspace{0.25cm} \noindent The \textit{write$\_$dump} command prints the final
positions of the atoms, and can be opened with VMD
to visualize the system.

\vspace{0.25cm} \noindent Run the \textit{input.lammps} file using LAMMPS. 

\vspace{0.25cm} \noindent [legend-to-add]Figure: Side view of the system. Periodic images are represented in darker
[legend-to-add]colors. Water molecules are in red and white, $\text{Na}^+$
[legend-to-add]ions in purple, $\text{Cl}^-$ ions in lime, and wall atoms in
[legend-to-add]gray. Note the absence of atomic defect at the cell boundaries.
[legend-to-add]See the corresponding \href{https://youtu.be/SK3FkJt0TmM}{video}.

\vspace{0.25cm} \noindent Always check that your system has been correctly created
by looking at the periodic images. Atomic defects may
occur at the boundary.

\subsection{Energy minimization}
\noindent \begin{tcolorbox}[colback=mylightblue!5!white,colframe=mylightblue!75!black,title=Why is energy minimization necessary?]

\vspace{0.25cm} \noindent It is clear from the way the system has been created that
the atoms are not at equilibrium distances from each
other. Indeed, some of the ions added using the \textit{create$\_$atoms}
commands are too close to the water molecules.
If we were to start a \textit{normal} (i.e. with a timestep of about 1 fs)
molecular dynamics simulation now, the atoms
would exert huge forces on each other, accelerate
brutally, and the simulation would likely fail.
\end{tcolorbox}

\noindent \begin{tcolorbox}[colback=mylightblue!5!white,colframe=mylightblue!75!black,title=Dealing with overlapping atoms]

\vspace{0.25cm} \noindent MD simulations failing due to overlapping atoms are
extremely common. If it occurs, you can either

\begin{itemize}
\item delete the overlapping atoms using the \textit{delete$\_$atoms} command of LAMMPS,
\item move the atoms to more reasonable distances before the simulation starts using energy minimization, or using molecular dynamics with a small timestep.
\end{itemize}
\end{tcolorbox}

\noindent Let us move the atoms and place them
in more energetically favorable positions before starting the simulation.
Let us call this step \textit{energy minimization}, although it is not 
a conventional \textit{minimization} as done for instance
in tutorial \hyperref[lennard-jones-label]{Lennard Jones fluid}.

\vspace{0.25cm} \noindent To perform this energy minimization, let us
create a new folder named \textit{minimization/} next to \textit{systemcreation/},
and create a new input file named \textit{input.lammps} in it. Copy the following lines
in \textit{input.lammps}:

\begin{lcverbatim}
boundary p p p
units real
atom_style full
bond_style harmonic
angle_style harmonic
pair_style lj/cut/tip4p/long 1 2 1 1 0.1546 12.0
kspace_style pppm/tip4p 1.0e-4

read_data ../systemcreation/system.data

include ../PARM.lammps
include ../GROUP.lammps
\end{lcverbatim}

\noindent The only difference with the previous input is that, instead
of creating a new box and new atoms, we open the
previously created file \textit{system.data} located in \textit{systemcreation/}.
The file \textit{system.data} contains the definition of the simulation box
and the positions of the atoms.

\vspace{0.25cm} \noindent Now, let us create a first simulation step using a relatively small 
timestep ($0.5\,\text{fs}$), as well as a low temperature
of $T = 1\,\text{K}$:

\begin{lcverbatim}
fix mynve fluid nve/limit 0.1
fix myber fluid temp/berendsen 1 1 100
fix myshk H2O shake 1.0e-4 200 0 b 1 a 1
timestep 0.5
\end{lcverbatim}

\noindent Just like \textit{fix nve}, the fix \textit{nve/limit} performs constant NVE integration to
update positions and velocities of the atoms at each
timestep, but also limits the maximum distance atoms can travel at
each timestep. Here, only the fluid molecules and ions will move.

\vspace{0.25cm} \noindent The \textit{fix temp/berendsen} rescales the
velocities of the atoms to force the temperature of the system
to reach the desired value of 1 K, and the shake algorithm
is used in order to maintain the shape of the water molecules.

\vspace{0.25cm} \noindent Let us also print the atom positions in a \textit{.lammpstrj} file by
adding the following line to \textit{input.lammps}:

\begin{lcverbatim}
dump mydmp all atom 1000 dump.lammpstrj
thermo 200
\end{lcverbatim}

\noindent Finally, let us run for 4000 steps. Add the 
following lines into \textit{input.lammps}:

\begin{lcverbatim}
run 4000
\end{lcverbatim}

\noindent In order to better equilibrate the system, let us perform 
two additional steps with a larger timestep and a larger
imposed temperature:

\begin{lcverbatim}
fix myber fluid temp/berendsen 300 300 100
timestep 1.0

run 4000

unfix mynve
fix mynve fluid nve

run 4000

write_data system.data
\end{lcverbatim}

\noindent For the last of the 3 steps, fix \textit{nve} is used instead of 
\textit{nve/limit}, which will allow for a better relaxation of the 
atom positions.

\vspace{0.25cm} \noindent When running the \textit{input.lammps} file with LAMMPS, you should see that the
the total energy of the system decreases during the first 
of the 3 steps, before re-increasing a little after the 
temperature is increased from 1 to $300\,\text{K}$.

\vspace{0.25cm} \noindent [legend-to-add]Figure: Energy as a function of time extracted from the log
[legend-to-add]file using \textit{Python} and \textit{lammps$\_$logfile}.

\vspace{0.25cm} \noindent If you look at the trajectory using VMD, you will see some
of the atoms, in particular, the ones that were
initially in problematic positions. 

\subsection{System equilibration}
\noindent Now, let us equilibrate further the entire system by letting both
fluid and piston relax at ambient temperature.

\vspace{0.25cm} \noindent Create a new folder called \textit{equilibration/} next to 
the previously created folders, and create a new
\textit{input.lammps} file in it. Add the following lines into \textit{input.lammps}:

\begin{lcverbatim}
boundary p p p
units real
atom_style full
bond_style harmonic
angle_style harmonic
pair_style lj/cut/tip4p/long 1 2 1 1 0.1546 12.0
kspace_style pppm/tip4p 1.0e-4

read_data ../minimization/system.data

include ../PARM.lammps
include ../GROUP.lammps
\end{lcverbatim}

\noindent Finally, let us complete the \textit{input.lammps} file:

\begin{lcverbatim}
fix mynve all nve
fix myber all temp/berendsen 300 300 100
fix myshk H2O shake 1.0e-4 200 0 b 1 a 1
fix myrct all recenter NULL NULL 0
timestep 1.0
\end{lcverbatim}

\noindent The fix \textit{recenter} has no influence on the dynamics, but will
keep the system in the center of the box, which makes the
visualization easier.

\vspace{0.25cm} \noindent Then, add the following lines to \textit{input.lammps} for
the trajectory visualization and output:

\begin{lcverbatim}
dump mydmp all atom 1000 dump.lammpstrj
thermo 500
variable walltopz equal xcm(walltop,z)
variable wallbotz equal xcm(wallbot,z)
variable deltaz equal v_walltopz-v_wallbotz
fix myat1 all ave/time 100 1 100 v_deltaz file interwall_distance.dat
\end{lcverbatim}

\noindent The first two variables extract the centers of mass of
the two walls. Then, the \textit{deltaz}
variable is used to calculate the distance between
the two variables \textit{walltopz}
and \textit{wallbotz}, i.e. the distance between the two walls.

\vspace{0.25cm} \noindent Finally, let us add the \textit{run} command: 

\begin{lcverbatim}
run 30000
write_data system.data  
\end{lcverbatim}

\noindent Run the \textit{input.lammps} file using LAMMPS.

\vspace{0.25cm} \noindent As seen from the data printed by \textit{fix myat1},
the distance $\delta_z$ between the two walls
reduces until it reaches an equilibrium value.

\vspace{0.25cm} \noindent [legend-to-add]Figure: Distance between the walls as a function of time.
[legend-to-add]After a few picoseconds, the distance between the two walls equilibrates near
[legend-to-add]its final value. 

\vspace{0.25cm} \noindent Note that it is generally recommended to run longer equilibration.
Here, for instance, the slowest
process in the system is probably the ionic diffusion. Therefore the equilibration 
should in principle be longer than the time
the ions need to diffuse over the size of the pore
($\approx 1.2\,\text{nm}$), i.e. of the order of half a nanosecond.

\section{Imposed shearing}
\noindent From the equilibrated configuration, let us impose a laterial
motion to the two walls and shear the electrolyte.
In a new folder called \textit{shearing/},
create a new \textit{input.lammps} file that starts like the previous ones:

\begin{lcverbatim}
boundary p p p
units real
atom_style full
bond_style harmonic
angle_style harmonic
pair_style lj/cut/tip4p/long 1 2 1 1 0.1546 12.0
kspace_style pppm/tip4p 1.0e-4
\end{lcverbatim}

\noindent Let us import the previously equilibrated data,
include the parameter and group files,
and then deal with the dynamics of the system.

\begin{lcverbatim}
read_data ../equilibration/system.data

include ../PARM.lammps
include ../GROUP.lammps

fix mynve all nve
compute Tfluid fluid temp/partial 0 1 1
fix myber1 fluid temp/berendsen 300 300 100
fix_modify myber1 temp Tfluid
compute Twall wall temp/partial 0 1 1
fix myber2 wall temp/berendsen 300 300 100
fix_modify myber2 temp Twall
fix myshk H2O shake 1.0e-4 200 0 b 1 a 1
fix myrct all recenter NULL NULL 0
\end{lcverbatim}

\noindent One difference here is that two thermostats are used,
one for the fluid (\textit{myber1}) and one
for the solid (\textit{myber2}). The use of \textit{fix$\_$modify} together
with \textit{compute} ensures that the right temperature value
is used by the thermostats.

\vspace{0.25cm} \noindent The use of temperature \textit{compute} with \textit{temp/partial 0 1 1}
is meant to exclude the \textit{x} coordinate from the
thermalization, which is important since a large velocity
will be imposed along \textit{x}. 

\vspace{0.25cm} \noindent Then, let us impose the velocity of the two walls 
by adding the following command to \textit{input.lammps}:

\begin{lcverbatim}
fix mysf1 walltop setforce 0 NULL NULL
fix mysf2 wallbot setforce 0 NULL NULL
velocity wallbot set -2e-4 NULL NULL
velocity walltop set 2e-4 NULL NULL
\end{lcverbatim}

\noindent The \textit{setforce} commands cancel the forces on \textit{walltop} and
\textit{wallbot}, respectively. Therefore the atoms of the two groups do not
experience any force from the rest of the system. In the absence of force
acting on those atoms, they will conserve their initial velocity.

\vspace{0.25cm} \noindent The \textit{velocity} commands act only once and impose
the velocity of the atoms of the groups \textit{wallbot}
and \textit{walltop}, respectively.

\vspace{0.25cm} \noindent Finally, let us dump the atom positions, extract the
velocity profiles using several \textit{ave/chunk} commands, extract the
force applied on the walls, and then run for $200\,\text{ps}$
Add the following lines to \textit{input.lammps}:

\begin{lcverbatim}
dump mydmp all atom 5000 dump.lammpstrj
thermo 500
thermo_modify temp Tfluid

compute cc1 H2O chunk/atom bin/1d z 0.0 1.0
compute cc2 wall chunk/atom bin/1d z 0.0 1.0
compute cc3 ions chunk/atom bin/1d z 0.0 1.0

fix myac1 H2O ave/chunk 10 15000 200000 &
cc1 density/mass vx file water.profile_1A.dat
fix myac2 wall ave/chunk 10 15000 200000 &
cc2 density/mass vx file wall.profile_1A.dat
fix myac3 ions ave/chunk 10 15000 200000 &
cc3 density/mass vx file ions.profile_1A.dat

fix myat1 all ave/time 10 100 1000 f_mysf1[1] f_mysf2[1] file forces.dat

timestep 1.0
run 200000
write_data system.data
\end{lcverbatim}

\noindent Here, a binning of $1\,\text{Å}$ is used. For smoother
profiles, you can reduce its value.

\vspace{0.25cm} \noindent The averaged velocity profile of the fluid 
can be plotted. As expected here, the velocity
of the fluid is found to increase linearly along $z$.

\vspace{0.25cm} \noindent [legend-to-add]Figure: Velocity profiles for water molecules, ions and walls
[legend-to-add]along the \textit{z} axis. The line is a linear fit assuming that 
[legend-to-add]the pore size is $h = 1.8\,\text{nm}$.

\vspace{0.25cm} \noindent [legend-to-add]Figure: Water density $\rho$ profile
[legend-to-add]along the \textit{z} axis.

\vspace{0.25cm} \noindent From the force applied by the fluid on the solid, one can
extract the stress within the fluid, which allows one to
measure its viscosity $\dot{\eta}$ 
according to \href{https://pure.tudelft.nl/ws/portalfiles/portal/89280267/PhysRevFluids.6.034303.pdf}{gravelle2021}:
$\eta = \tau / \dot{\gamma}$ where $\tau$
is the stress applied by the fluid on the shearing wall, and
$\dot{\gamma}$ the shear rate (which is imposed
here) \cite{gravelle2021violations}. Here the shear rate
is approximatively $\dot{\gamma} = 16 \cdot 10^9\,\text{s}^{-1}$,
and using a surface area of $A = 6 \cdot 10^{-18}\,\text{m}^2$, one
gets an estimate for the shear viscosity for the confined
fluid of $\eta = 6.6\,\text{mPa.s}$

\vspace{0.25cm} \noindent The viscosity calculated at such a high shear rate may
differ from the expected \textit{bulk} value. In general, it is recommended to use a lower
value for the shear rate. Note that for lower shear rates, the ratio of noise-to-signal
is larger, and longer simulations are needed.

\vspace{0.25cm} \noindent Another important point is that the viscosity of a fluid next to a solid surface is
typically larger than in bulk due to interaction with the
walls. Therefore, one expects the present simulation to return 
a viscosity that is slightly larger than what would
be measured in the absence of a wall.

\vspace{0.25cm} \noindent You can access the input scripts and data files that
are used in these tutorials from \href{https://github.com/lammpstutorials/lammpstutorials-inputs/}{this Github repository}.
This repository also contains the full solutions to the exercises.

\section{Going further with exercises}
\noindent Each exercise comes with a proposed solution, 
see \hyperref[solutions-label]{Solutions to the exercises}.

\subsection{Induce a Poiseuille flow}
\noindent Instead of inducing a shearing of the fluid using the walls,
induce a net flux of the liquid in the direction tangential
to the walls. The walls must be kept immobile.

\vspace{0.25cm} \noindent Extract the velocity profile, and make sure that the
resulting velocity profile is consistent with the Poiseuille equation,
which can be derived from the Stokes equation $\eta \nabla \textbf{v} = - \textbf{f} \rho$
where $f$ is the applied force,
$\rho$ is the fluid density,
$\eta$ is the fluid viscosity.

\vspace{0.25cm} \noindent [legend-to-add]Figure: Velocity profiles of the water molecules along the \textit{z} axis (disks).
[legend-to-add]The line is the Poiseuille equation.

\vspace{0.25cm} \noindent An important step is to choose the proper value for the additional force.


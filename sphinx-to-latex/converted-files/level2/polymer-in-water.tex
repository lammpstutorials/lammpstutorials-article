\chapter{Polymer in water}
\label{all-atoms-label}

\noindent \vspace{-1cm} \noindent \textcolor{graytitle}{\textit{{\Large Solvating and stretching a small polymer molecule}}\vspace{0.5cm} }

\vspace{0.25cm} \noindent The goal of this tutorial is to use LAMMPS and solvate a small
hydrophilic polymer (PEG - PolyEthylene Glycol) in a reservoir of water. 

\vspace{0.25cm} \noindent An all-atom description is used for both PEG (GROMOS 54A7 force
field \cite{schmid2011definition}) and water
(SPC flexible model \cite{wu2006flexible}) and the long
range Coulomb interactions are solved using the PPPM solver \cite{luty1996calculating}.    
Once the water reservoir is properly
equilibrated at the desired temperature and pressure, the polymer molecule
is added and a constant stretching force is applied to both
ends of the polymer. The evolution of the polymer length
is measured as a function of time.

\vspace{0.25cm} \noindent This tutorial was inspired by a \href{https://doi.org/10.1021/acsnano.6b07071}{publication} by Liese and coworkers, in which
molecular dynamics simulations are
compared with force spectroscopy experiments \cite{liese2017hydration}.

\vspace{0.25cm} \noindent If you are completely new to LAMMPS, I recommend that
you follow this tutorial on a simple \hyperref[lennard-jones-label]{Lennard Jones fluid} first.

\section{Preparing the water reservoir}
\noindent In this tutorial, the water reservoir is first prepared in the absence of
polymer. A rectangular box of water is created and
equilibrated at ambient temperature and ambient pressure.
The SPC/Fw water model is used \cite{wu2006flexible}, which is
a flexible variant of the rigid SPC (simple point charge)
model \cite{berendsen1981interaction}.

\vspace{0.25cm} \noindent Create a folder named \textit{pureH2O/}. Inside this folder, create
an empty text file named \textit{input.lammps}. Copy the following
lines in it:

\begin{lcverbatim}
units real
atom_style full
bond_style harmonic
angle_style harmonic
dihedral_style harmonic
pair_style lj/cut/coul/long 12
kspace_style pppm 1e-5
special_bonds lj 0.0 0.0 0.5 coul 0.0 0.0 1.0 angle yes
\end{lcverbatim}

\noindent With the unit style \textit{real}, masses are in grams per
mole, distances in Ångstroms, time in femtoseconds, energies
in Kcal/mole. With the \textit{atom$\_$style full}, each atom is a dot
with a mass and a charge that can be
linked by bonds, angles, dihedrals and/or impropers. The \textit{bond$\_$style},
\textit{angle$\_$style}, and \textit{dihedral$\_$style} commands define the
potentials for the bonds, angles, and dihedrals used in the simulation,
here \textit{harmonic}.

\vspace{0.25cm} \noindent Always refer to the LAMMPS \href{https://docs.lammps.org/}{documentation} if you have doubts about the
potential used by LAMMPS. For instance, this \href{https://docs.lammps.org/angle_harmonic.html}{page}
gives the expression for the harmonic angular potential.

\vspace{0.25cm} \noindent Finally, the \textit{special$\_$bonds} command cancels the
Lennard-Jones interactions between the closest
atoms of the same molecule.

\begin{tcolorbox}[colback=mylightblue!5!white,colframe=mylightblue!75!black,title=About *special bonds*]

\vspace{0.25cm} \noindent Usually, molecular dynamics force fields are parametrized assuming that
the first neighbors within a molecule do not
interact directly through LJ or Coulomb potential. Here, since we
use \textit{lj 0.0 0.0 0.5} and \textit{coul 0.0 0.0 1.0}, the first and second
neighbors in a molecule only interact through direct bond interactions.
For the third neighbor (here third neighbor only concerns the PEG molecule,
not the water), only half of the LJ interaction will be taken into account,
and the full Coulomb interaction will be used.   
\end{tcolorbox}

\noindent With the \textit{pair$\_$style} named \textit{lj/cut/coul/long}, atoms
interact through both a Lennard-Jones (LJ) potential and
Coulombic interactions. The value of $12\,\text{Å}$ is 
the cutoff.

\begin{tcolorbox}[colback=mylightblue!5!white,colframe=mylightblue!75!black,title=About cutoff in molecular dynamics]

\vspace{0.25cm} \noindent The cutoff of $12\,\text{Å}$ applies to both LJ and Coulombic
interactions, but in a different way. For LJ \textit{cut}
interactions, atoms interact with each other only if they
are separated by a distance smaller than the cutoff. For
Coulombic \textit{long}, interactions between atoms closer than
the cutoff are computed directly, and interactions between
atoms outside that cutoff are computed in the reciprocal space.
\end{tcolorbox}

\noindent Finally, the \textit{kspace} command defines the long-range solver for the (long)
Coulombic interactions. The \textit{pppm} style refers to
particle-particle particle-mesh.

\begin{tcolorbox}[colback=mylightblue!5!white,colframe=mylightblue!75!black,title=About PPPM]

\vspace{0.25cm} \noindent Extracted from \href{https://doi.org/10.1021/jp9518623}{Luty and van Gunsteren}:
The PPPM method is based on separating the total interaction
between particles into the sum of short-range
interactions, which are computed by direct
particle-particle summation, and long-range interactions,
which are calculated by solving Poisson's equation using
periodic boundary conditions (PBCs) :cite:`luty1996calculating`.
\end{tcolorbox}

\noindent Then, let us create a 3D simulation box of dimensions $9 \times 3 \times 3 \; \text{nm}^3$,
and make space for 9 atom types (2 for
the water molecule, and 7 for the polymer molecule), 7 bond types, 8
angle types, and 4 dihedral types.
Copy the following lines into \textit{input.lammps}:

\begin{lcverbatim}
region box block -45 45 -15 15 -15 15
create_box 9 box &
bond/types 7 &
angle/types 8 &
dihedral/types 4 &
extra/bond/per/atom 3 &
extra/angle/per/atom 6 &
extra/dihedral/per/atom 10 &
extra/special/per/atom 14
\end{lcverbatim}

\noindent \begin{tcolorbox}[colback=mylightblue!5!white,colframe=mylightblue!75!black,title=About extra per atom commands]

\vspace{0.25cm} \noindent The \textit{extra/x/per/atom} commands are here for
memory allocation. These commands ensure that enough memory space is left for a
certain number of attributes for each atom. We won't worry
about those commands in this tutorial, just keep that in mind if one day
you see the following error
message \textit{ERROR: Molecule topology/atom exceeds system topology/atom}.
\end{tcolorbox}

\noindent Let us create a \textit{PARM.lammps} file containing all the
parameters (masses, interaction energies, bond equilibrium
distances, etc). In \textit{input.lammps}, add the following line:

\begin{lcverbatim}
include ../PARM.lammps
\end{lcverbatim}

\noindent Then, download and save the \href{https://lammpstutorials.github.io/lammpstutorials-inputs/level2/polymer-in-water/PARM.lammps}{parameter} file
next to the \textit{pureH2O/} folder.

\vspace{0.25cm} \noindent Within \textit{PARM.lammps}, the \textit{mass} and \textit{pair$\_$coeff} of atoms
of types 8 and 9 are for water and the 
atoms of types 1 to 7 are for the polymer
molecule. Similarly, the \textit{bond$\_$coeff 7} and 
\textit{angle$\_$coeff 8} are for water, while all
the other parameters are for the polymer.

\vspace{0.25cm} \noindent Let us create water molecules. To do so, let us
define what a water molecule is using a molecule \textit{template} called
\textit{H2O-SPCFw.mol}, and then randomly create 1050 molecules.
Add the following lines into \textit{input.lammps}:

\begin{lcverbatim}
molecule h2omol H2O-SPCFw.mol
create_atoms 0 random 1050 87910 NULL mol &
    h2omol 454756 overlap 1.0 maxtry 50
\end{lcverbatim}

\noindent The \textit{overlap 1} option of the \textit{create$\_$atoms} command ensures that no atoms are
placed exactly in the same position, as this would cause the simulation to
crash. The \textit{maxtry 50} asks LAMMPS to try at most
50 times to insert the molecules, which is useful in case some
insertion attempts are rejected due to overlap. In some cases, depending on
the system and the values of \textit{overlap}
and \textit{maxtry}, LAMMPS may not create the desired number of molecules.
Always check the number of created atoms in the \textit{log} file after
starting the simulation:

\begin{lcverbatim}
Created 1050 atoms
\end{lcverbatim}

\noindent When LAMMPS fails to create the desired number of molecules, a WARNING
appears in the \textit{log} file.

\vspace{0.25cm} \noindent The molecule template named \textit{H2O-SPCFw.mol}
can be \href{https://lammpstutorials.github.io/lammpstutorials-inputs/level2/polymer-in-water/pureH2O/H2O-SPCFw.mol}{downloaded}
and saved in the \textit{pureH2O/} folder.
This template contains the necessary structural
information of a water molecule, such as the number of atoms, 
the id of the atoms that are connected by bonds, by angles, etc.

\vspace{0.25cm} \noindent Then, let us organize the atoms of types 8 and 9 of the water molecules
in a group named \textit{H2O} and perform a small energy minimization. The
energy minimization is mandatory here given the small \textit{overlap} value
of 1 Ångstrom chosen in the \textit{create$\_$atoms} command. Add the following lines
to \textit{input.lammps}:

\begin{lcverbatim}
group H2O type 8 9
minimize 1.0e-4 1.0e-6 100 1000
reset_timestep 0
\end{lcverbatim}

\noindent The \textit{reset$\_$timestep} command is optional. It is used here 
because the \textit{minimize} command is usually performed over an arbitrary
number of steps.

\vspace{0.25cm} \noindent Let us use the \textit{fix npt} to
control the temperature of the molecules with a Nosé-Hoover thermostat and
the pressure of the system with a Nosé-Hoover barostat 
\cite{nose1984unified, hoover1985canonical, martyna1994constant},
by adding the following line to \textit{input.lammps}:

\begin{lcverbatim}
fix mynpt all npt temp 300 300 100 iso 1 1 1000
\end{lcverbatim}

\noindent The \textit{fix npt} allows us to impose both a temperature of $300\,\text{K}$
(with a damping constant of $100\,\text{fs}$),
and a pressure of 1 atmosphere (with a damping constant of $1000\,\text{fs}$).
With the \textit{iso} keyword, the three dimensions of the box will be re-scaled
simultaneously.

\vspace{0.25cm} \noindent Let us print the atom positions in a \textit{.lammpstrj} file every 1000
steps (i.e. 1 ps), print the temperature volume, and
density every 100 steps in 3 separate data files, and
print the information in the terminal every 1000 steps:

\begin{lcverbatim}
dump mydmp all atom 1000 dump.lammpstrj
variable mytemp equal temp
variable myvol equal vol
fix myat1 all ave/time 10 10 100 v_mytemp file temperature.dat
fix myat2 all ave/time 10 10 100 v_myvol file volume.dat
variable myoxy equal count(H2O)/3
variable mydensity equal ${myoxy}/v_myvol
fix myat3 all ave/time 10 10 100 v_mydensity file density.dat
thermo 1000
\end{lcverbatim}

\noindent The variable \textit{myoxy} corresponds to the number of atoms
divided by 3, i.e. the number of molecules.

\begin{tcolorbox}[colback=mylightblue!5!white,colframe=mylightblue!75!black,title=On calling variables in LAMMPS]

\vspace{0.25cm} \noindent Both dollar sign and underscore can be used to call a previously defined
variable. With the dollar sign, the initial value of the variable is returned,
while with the underscore, the instantaneous value of the variable is returned. 
To probe the temporal evolution of a variable with time,
the underscore must be used.
\end{tcolorbox}

\noindent Finally, let us set the timestep to 1.0 fs,
and run the simulation for 20 ps by adding the
following lines to \textit{input.lammps}:

\begin{lcverbatim}
timestep 1.0
run 20000

write_data H2O.data
\end{lcverbatim}

\noindent The final state is written into \textit{H2O.data}.

\vspace{0.25cm} \noindent If you open the \textit{dump.lammpstrj} file using VMD, you should
see the system quickly reaching its equilibrium volume and density.

\vspace{0.25cm} \noindent [legend-to-add]Figure: Water reservoir after equilibration. Oxygen atoms are in red, and
[legend-to-add]hydrogen atoms are in white.

\vspace{0.25cm} \noindent You can also open the \textit{density.dat} file to ensure that the system converged
toward an equilibrated liquid water system during the 20 ps of simulation.

\vspace{0.25cm} \noindent [legend-to-add]Figure: Evolution of the density of water with time. The
[legend-to-add]density $\rho$ reaches
[legend-to-add]a plateau after $\approx 10\,\text{ps}$.

\vspace{0.25cm} \noindent If needed, you can \href{https://lammpstutorials.github.io/lammpstutorials-inputs/level2/polymer-in-water/pureH2O/H2O.data}{download} the water reservoir I have
equilibrated and use it to continue with the tutorial.

\section{Solvating the PEG in water}
\noindent Once the water reservoir is equilibrated, we can safely
include the PEG polymer in the water before performing the pull experiment
on the polymer.

\vspace{0.25cm} \noindent The PEG molecule topology was downloaded from the \href{https://atb.uq.edu.au/}{ATB}
repository \cite{malde2011automated, oostenbrink2004biomolecular}.
It has a formula $\text{C}_{28}\text{H}_{58}\text{O}_{15}$,
and the parameters are taken from
the GROMOS 54A7 force field \cite{schmid2011definition}.

\vspace{0.25cm} \noindent [legend-to-add]Figure: The PEG molecule in vacuum. The carbon atoms are in gray,
[legend-to-add]the oxygen atoms in red, and the hydrogen atoms in white.

\vspace{0.25cm} \noindent Create a second folder alongside \textit{pureH2O/}
and call it \textit{mergePEGH2O/}. Create a new blank file in it,
call it \textit{input.lammps}. Within \textit{input.lammps}, copy the same first lines as
previously:

\begin{lcverbatim}
units real
atom_style full
bond_style harmonic
angle_style harmonic
dihedral_style harmonic
pair_style lj/cut/coul/long 12
kspace_style pppm 1e-5
special_bonds lj 0.0 0.0 0.5 coul 0.0 0.0 1.0 angle yes dihedral yes
\end{lcverbatim}

\noindent Then, import the previously generated data file \textit{H2O.data}
as well as the \textit{PARM.lammps} file:

\begin{lcverbatim}
read_data ../pureH2O/H2O.data &
    extra/bond/per/atom 3 &
    extra/angle/per/atom 6 &
    extra/dihedral/per/atom 10 &
    extra/special/per/atom 14
include ../PARM.lammps
\end{lcverbatim}

\noindent Let us create a molecule called \textit{pegmol} from
the molecule \href{https://lammpstutorials.github.io/lammpstutorials-inputs/level2/polymer-in-water/mergePEGH2O/PEG-GROMOS.mol}{template}
for the PEG molecule, and let us create a single molecule in the middle of
the box:

\begin{lcverbatim}
molecule pegmol PEG-GROMOS.mol
create_atoms 0 single 0 0 0 mol pegmol 454756
\end{lcverbatim}

\noindent Let us create 2 groups to differentiate the PEG from the H2O,
by adding the following lines to \textit{input.lammps}:

\begin{lcverbatim}
group H2O type 8 9
group PEG type 1 2 3 4 5 6 7
\end{lcverbatim}

\noindent Water molecules that are overlapping with the PEG must be deleted to avoid
future crashing. Add the following line to \textit{input.lammps}:

\begin{lcverbatim}
delete_atoms overlap 2.0 H2O PEG mol yes
\end{lcverbatim}

\noindent Here, the value of 2 Ångstroms for the overlap cutoff was fixed arbitrarily
and can be chosen through trial and error. If the cutoff is too small, the 
simulation will crash. If the cutoff is too large, too many water molecules
will unnecessarily be deleted.

\vspace{0.25cm} \noindent Finally, let us use the \textit{fix npt} to control the temperature, as well as
the pressure by allowing the box size to be rescaled along the \textit{x} axis:

\begin{lcverbatim}
fix mynpt all npt temp 300 300 100 x 1 1 1000
timestep 1.0
\end{lcverbatim}

\noindent Once more, let us dump the atom positions as well as the system temperature
and volume:

\begin{lcverbatim}
dump mydmp all atom 100 dump.lammpstrj
thermo 100
variable mytemp equal temp
variable myvol equal vol
fix myat1 all ave/time 10 10 100 v_mytemp file temperature.dat
fix myat2 all ave/time 10 10 100 v_myvol file volume.dat
\end{lcverbatim}

\noindent Let us also print the total enthalpy:

\begin{lcverbatim}
variable myenthalpy equal enthalpy
fix myat3 all ave/time 10 10 100 v_myenthalpy file enthalpy.dat
\end{lcverbatim}

\noindent Finally, let us perform a short equilibration and print the
final state in a data file. Add the following lines to the data file:

\begin{lcverbatim}
run 30000
write_data mix.data
\end{lcverbatim}

\noindent If you open the \textit{dump.lammpstrj} file using VMD, 
or have a look at the evolution of the volume in \textit{volume.dat},
you should see that the box dimension slightly evolves along \textit{x}
to accommodate the new configuration.

\vspace{0.25cm} \noindent [legend-to-add]Figure: A single PEG molecule in water. Water molecules are represented as
[legend-to-add]a transparent continuum field for clarity.

\section{Stretching the PEG molecule}
\noindent Here, a constant forcing is applied to the two ends of the PEG molecule
until it stretches. Create a new folder next to the previously created
folders, call it \textit{pullonPEG/}, and create a new input file in it
called \textit{input.lammps}.

\vspace{0.25cm} \noindent First, let us create a variable \textit{f0} corresponding to the magnitude
of the force we are going to apply. The force magnitude is
chosen to be large enough to overcome the thermal
agitation and the entropic contribution from both water
and PEG molecules (it was chosen by trial and error). Copy
in the \textit{input.lammps} file:

\begin{lcverbatim}
variable f0 equal 5
\end{lcverbatim}

\noindent Note that $1\,\text{kcal/mol/Å}$ corresponds
to $67.2\,\text{pN}$.
Then, copy the same lines as previously:

\begin{lcverbatim}
units real
atom_style full
bond_style harmonic
angle_style harmonic
dihedral_style harmonic
pair_style lj/cut/coul/long 12
kspace_style pppm 1e-5
special_bonds lj 0.0 0.0 0.5 coul 0.0 0.0 1.0 angle yes dihedral yes
\end{lcverbatim}

\noindent Start the simulation from the equilibrated PEG-water system and include
again the parameter file by adding the following lines to the \textit{input.lammps}:

\begin{lcverbatim}
read_data ../mergePEGH2O/mix.data
include ../PARM.lammps
\end{lcverbatim}

\noindent Then, let us create 4 atom groups: H2O and PEG (as previously), as well
as 2 groups containing only the 2 oxygen atoms of types 6 and 7,
respectively. Atoms of types 6 and 7 correspond to the oxygen atoms
located at the ends of the PEG molecule, which we are going to use to pull
on the PEG molecule. Add the following lines to the \textit{input.lammps}:

\begin{lcverbatim}
group H2O type 8 9
group PEG type 1 2 3 4 5 6 7
group topull1 type 6
group topull2 type 7
\end{lcverbatim}

\noindent Add the following \textit{dump} command to the input to print the atom positions
every 1000 steps:

\begin{lcverbatim}
dump mydmp all atom 1000 dump.lammpstrj
\end{lcverbatim}

\noindent Let us use a single Nosé-Hoover thermostat applied to all the atoms by
adding the following lines to \textit{input.lammps}:

\begin{lcverbatim}
timestep 1.0
fix mynvt all nvt temp 300 300 100
\end{lcverbatim}

\noindent Let us also print the end-to-end distance of the PEG,
here defined as the distance between the groups \textit{topull1}
and \textit{topull2}, as well as the temperature of the system 
by adding the following lines to \textit{input.lammps}:

\begin{lcverbatim}
variable mytemp equal temp
fix myat1 all ave/time 10 10 100 v_mytemp file output-temperature.dat
variable x1 equal xcm(topull1,x)
variable x2 equal xcm(topull2,x)
variable y1 equal xcm(topull1,y)
variable y2 equal xcm(topull2,y)
variable z1 equal xcm(topull1,z)
variable z2 equal xcm(topull2,z)
variable delta_r equal sqrt((v_x1-v_x2)^2+(v_y1-v_y2)^2+(v_z1-v_z2)^2)
fix myat2 all ave/time 10 10 100 v_delta_r &
    file output-end-to-end-distance.dat
thermo 1000
\end{lcverbatim}

\noindent Finally, let us simulate 30 picoseconds without any external forcing:

\begin{lcverbatim}
run 30000
\end{lcverbatim}

\noindent This first run serves as a benchmark to quantify the changes
induced by the forcing. Then, let us apply a forcing on the 2 oxygen
atoms using two \textit{add$\_$force} commands, and run for an extra 30 ps:

\begin{lcverbatim}
fix myaf1 topull1 addforce ${f0} 0 0
fix myaf2 topull2 addforce -${f0} 0 0
run 30000
\end{lcverbatim}

\noindent If you open the \textit{dump.lammpstrj} file using \textit{VMD}, you should
see that the PEG molecule eventually aligns in the direction
of the force.

\vspace{0.25cm} \noindent [legend-to-add]Figure: PEG molecule stretched along the \textit{x} direction in water.
[legend-to-add]Water molecules are represented as a transparent continuum 
[legend-to-add]field for clarity. See the corresponding \href{https://youtu.be/mjc6O6d9F-Y}{video}.

\vspace{0.25cm} \noindent The evolution of the end-to-end distance over time
shows the PEG adjusting to the external forcing:

\vspace{0.25cm} \noindent [legend-to-add]Figure: Evolution of the end-to-end distance of the PEG molecule
[legend-to-add]with time. The forcing starts at $t = 30$ ps.

\vspace{0.25cm} \noindent There is a follow-up to this polymer in water tutorial as \hyperref[mda-label]{MDAnalysis tutorial},
where the trajectory is imported in Python using MDAnalysis.

\vspace{0.25cm} \noindent You can access the input scripts and data files that
are used in these tutorials from \href{https://github.com/lammpstutorials/lammpstutorials-inputs/}{this Github repository}.
This repository also contains the full solutions to the exercises.

\section{Going further with exercises}
\noindent Each exercise comes with a proposed solution, 
see \hyperref[solutions-label]{Solutions to the exercises}.

\subsection{Extract the radial distribution function}
\noindent Extract the radial distribution functions (RDF or $g(r)$)
between the oxygen atom of the water molecules
and the oxygen atom from the PEG molecule. Compare the rdf
before and after the force is applied to the PEG.

\vspace{0.25cm} \noindent [legend-to-add]Figure: Radial distribution function between the oxygen atoms 
[legend-to-add]of water, as well as between the oxygen atoms of water and the 
[legend-to-add]oxygen atoms of the PEG molecule.  

\vspace{0.25cm} \noindent Note the difference in the structure of the water before and after
the PEG molecule is stretched. This effect is described in
the 2017 publication by Liese et al. \cite{liese2017hydration}.

\subsection{Add salt to the system}
\noindent Realistic systems usually contain ions. Let us add some $\text{Na}^+$ and 
$\text{Cl}^-$ ions to our current PEG-water system.

\vspace{0.25cm} \noindent Add some $\text{Na}^+$ and 
$\text{Cl}^-$ ions to the mixture using the method
of your choice. $\text{Na}^+$ ions are 
characterised by their mass $m = 22.98\,\text{g/mol}$,
their charge $q = +1\,e$, and Lennard-Jones
parameters, $\epsilon = 0.0469\,\text{kcal/mol}$
and $\sigma = 0.243\,\text{nm}$,
and $\text{Cl}^-$ ions by their
mass $m = 35.453\,\text{g/mol}$,
charge $q = -1\,e$ and Lennard-Jones
parameters, $\epsilon = 0.15\,\text{kcal/mol}$,
and $\sigma = 0.4045\,\text{nm}$.

\vspace{0.25cm} \noindent [legend-to-add]Figure: A PEG molecule in the electrolyte with $\text{Na}^+$ ions in 
[legend-to-add]purple and $\text{Cl}^-$ ions in cyan.

\subsection{Evaluate the deformation of the PEG}
\noindent Once the PEG is fully stretched, its structure differs from the
unstretched case. The deformation can be probed by extracting the typical
intra-molecular parameters, such as the typical angles of the dihedrals.

\vspace{0.25cm} \noindent Extract the histograms of the angular distribution of the PEG dihedrals
in the absence and the presence of stretching.

\vspace{0.25cm} \noindent [legend-to-add]Figure: Probability distribution for the dihedral angle $\phi$, for a stretched
[legend-to-add]and for an unstretched PEG molecule.


\chapter{Reactive silicon dioxide}
\label{reactive-silicon-dioxide-label}

\noindent \vspace{-1cm} \noindent \textcolor{graytitle}{\textit{{\Large Simulating a chemically reactive structure}}\vspace{0.5cm} }

\vspace{0.25cm} \noindent The objective of this tutorial is to use a 
reactive force field (\textit{ReaxFF} \cite{van2001reaxff, zou2012investigation}),
and calculate the partial charges of a system undergoing
deformation, as well as chemical bond formation and breaking.  

\vspace{0.25cm} \noindent The system simulated here is a block of silicon dioxide (SiO2), that is deformed 
until rupture. Particular attention is given to the evolution of the charges
of the atoms during the deformation of the structure, and 
the chemical reactions occurring due to the deformation
are tracked.

\vspace{0.25cm} \noindent If you are completely new to LAMMPS, I recommend that
you follow this tutorial on a simple \hyperref[lennard-jones-label]{Lennard Jones fluid} first.

\section{Prepare and relax}
\noindent Create a folder, name it \textit{RelaxSilica/},
and \href{https://lammpstutorials.github.io/lammpstutorials-inputs/level3/reactive-silicon-dioxide/RelaxSilica/silica.data}{download} the initial topology of a small
amorphous silica structure.

\begin{tcolorbox}[colback=mylightblue!5!white,colframe=mylightblue!75!black,title=About the initial structure]

\vspace{0.25cm} \noindent The system was created by temperature annealing using another force field 
named \href{https://lammpstutorials.github.io/lammpstutorials-inputs/level3/reactive-silicon-dioxide/CreateSilica/SiO.1990.vashishta}{vashishta}. In case you are
interested in the input creation, the files
used for creating the initial topology is available
\href{https://lammpstutorials.github.io/lammpstutorials-inputs/level3/reactive-silicon-dioxide/CreateSilica/input.lammps}{here}.
\end{tcolorbox}

\noindent If you open the \textit{silica.data} file, you can see 
by looking at the Atoms section that
all silicon atoms have the same charge $q = 1.1\,\text{e}$,
and all oxygen atoms the charge $q = -0.55\,\text{e}$.
This is common with classical force field and will change once
\textit{ReaxFF} is used. Let us keep that in mind for now.

\vspace{0.25cm} \noindent The first step we need to perform here is to relax
the structure with \textit{ReaxFF}, which we are gonna do using molecular
dynamics. To make sure that the system equilibrates
nicely, let us track some changes over time.

\vspace{0.25cm} \noindent Create an input file called \textit{input.lammps} in \textit{RelaxSilica/},
and copy the following lines in it: 

\begin{lcverbatim}
units real
atom_style full

read_data silica.data

mass 1 28.0855 # Si
mass 2 15.999 # O
\end{lcverbatim}

\noindent So far, the input is very similar to what was seen
in the previous tutorials. Some basic parameters are
defined (\textit{units}, \textit{atom$\_$style} and \textit{masses}), and 
the \textit{.data} file is imported by the \textit{read$\_$data} command.
Now let us enter 3 crucial lines in the \textit{input.lammps} file:

\begin{lcverbatim}
pair_style reaxff NULL safezone 3.0 mincap 150
pair_coeff * * reaxCHOFe.ff Si O
fix myqeq all qeq/reaxff 1 0.0 10.0 1.0e-6 reaxff maxiter 400
\end{lcverbatim}

\noindent Here, the \textit{ReaxFF pair$\_$style} is used with no control file.
The \textit{safezone} and \textit{mincap} keywords have been added
to avoid memory allocation issues, which sometimes can trigger
the segmentation faults and \textit{bondchk} failed errors.

\vspace{0.25cm} \noindent The \textit{pair$\_$coeff} uses
the \href{https://lammpstutorials.github.io/lammpstutorials-inputs/level3/reactive-silicon-dioxide/RelaxSilica/reaxCHOFe.ff}{reaxCHOFe.ff} file which is assumed
to be saved within \textit{RelaxSilica/}.
For consistency, the atoms of type 1 are set as silicon (Si),
and the atoms of type 2 as oxygen (O).

\vspace{0.25cm} \noindent Finally, the \textit{fix qeq/reaxff} is used to perform charge equilibration. The charge
equilibration occurs at every step. The values 0.0 and 10.0
are the low and the high cutoffs, respectively, and $1.0 \text{e} -6$ is a
tolerance. Finally, \textit{maxiter} sets an upper limit to the number of attempts to
equilibrate the charge. 

\begin{tcolorbox}[colback=mylightblue!5!white,colframe=mylightblue!75!black,title=Note]

\vspace{0.25cm} \noindent If the charge does not
properly equilibrate despite the 400 attempts, a warning will appear. Such warnings
are likely to appear at the beginning of the simulation if the initial charges
are too far from the equilibrium values.
\end{tcolorbox}

\noindent Then, let us add some commands to the \textit{input.lammps} file 
to measure the evolution of the charges during the simulation:

\begin{lcverbatim}
group grpSi type 1
group grpO type 2
variable qSi equal charge(grpSi)/count(grpSi)
variable qO equal charge(grpO)/count(grpO)
\end{lcverbatim}

\noindent Let us also print the charge in the \textit{.log} file by using \textit{thermo$\_$style},
and create a \textit{.lammpstrj} file for visualization.
Add the following lines to the \textit{input.lammps}:

\begin{lcverbatim}
thermo 5
thermo_style custom step temp etotal press vol v_qSi v_qO
dump dmp all custom 100 dump.lammpstrj id type q x y z
\end{lcverbatim}

\noindent Let us also use the \textit{fix reaxff/species} to evaluate what
species are present within the simulation. It will
be useful later when the system is deformed:

\begin{lcverbatim}
fix myspec all reaxff/species 5 1 5 species.log element Si O
\end{lcverbatim}

\noindent Here, the information will be printed every 5 steps in a
file named \textit{species.log}.

\vspace{0.25cm} \noindent Let us perform a very short run using the anisotropic NPT command
and relax the density of the system. 

\begin{lcverbatim}
velocity all create 300.0 3482028
fix mynpt all npt temp 300.0 300.0 100 aniso 1.0 1.0 1000
timestep 0.5

run 5000

write_data silica-relaxed.data
\end{lcverbatim}

\noindent Run the \textit{input.lammps} file using LAMMPS. As can be seen from \textit{species.log},
only one species is detected, called \textit{Si192O384}, which is the entire system.

\vspace{0.25cm} \noindent As the simulation progresses, you can see that the charges of the atoms are fluctuating
since the charge of every individual atom is adjusting to its local environment.

\vspace{0.25cm} \noindent [legend-to-add]Figure: Average charge per atom of the silicon (a) and oxygen (b)
[legend-to-add]atoms during equilibration, as given by the
[legend-to-add]\textit{v$\_$qSi} and \textit{v$\_$qO} variables.

\vspace{0.25cm} \noindent One can see that the charges of the atoms are strongly fluctuating
at the beginning of the simulation. This early fluctuation correlates
with a rapid volume change of the box, during which
the inter-atomic distances are expected to quickly change.

\vspace{0.25cm} \noindent [legend-to-add]Figure: Volume of the system as a function of time.

\vspace{0.25cm} \noindent Since each atom has a charge that depends on its local environment,
the charge values are expected to be different for every atom in the system.
We can plot the charge distribution $P(q)$, using the charge values
printed in the \textit{.lammptrj} file. 

\vspace{0.25cm} \noindent [legend-to-add]Figure: Probability distribution of charge of silicon (positive, blue)
[legend-to-add]and oxygen (negative, orange) atoms during equilibration.

\vspace{0.25cm} \noindent Using VMD and coloring the atoms by their charges, one can see that 
the atoms with the extreme-most charges are located at defects in the 
amorphous structure (here at the positions of the dandling oxygen groups).

\vspace{0.25cm} \noindent [legend-to-add]Figure: A slice of the amorphous silica, where atoms are colored by their
[legend-to-add]charges. Dandling oxygen groups appear in greenish, bulk Si atoms with a
[legend-to-add]charge of about 1.8e appear in red/orange, and bulk O atoms with a charge of
[legend-to-add]about $-0.9\text{e}$ appear in blue. To color the atoms by their
[legend-to-add]charge using VMD, use \textit{Charge} as the coloring method in the representation
[legend-to-add]windows, and then tune the \textit{Color scale} in the \textit{Color control windows}.

\section{Deform the structure}
\noindent Let us apply a deformation to the structure to force some
$\text{Si}-\text{O}$ bonds to break and re-assemble. 

\vspace{0.25cm} \noindent Next to \textit{RelaxSilica/}, create a folder, call it \textit{Deform/} and create a
file named \textit{input.lammps} in it. Copy the same lines as previously in
\textit{input.lammps}:

\begin{lcverbatim}
units real
atom_style full

read_data ../RelaxSilica/silica-relaxed.data

mass 1 28.0855 # Si
mass 2 15.999 # O

pair_style reaxff NULL safezone 3.0 mincap 150
pair_coeff * * ../RelaxSilica/reaxCHOFe.ff Si O
fix myqeq all qeq/reaxff 1 0.0 10.0 1.0e-6 reaxff maxiter 400
\end{lcverbatim}

\noindent The only differences with the previous \textit{input.lammps} file are the
paths to the \textit{.data} and \textit{.ff} files located within \textit{RelaxSilica/}.
Copy the following lines as well:

\begin{lcverbatim}
group grpSi type 1
group grpO type 2
variable qSi equal charge(grpSi)/count(grpSi)
variable qO equal charge(grpO)/count(grpO)

thermo 5
thermo_style custom step temp etotal press vol v_qSi v_qO
dump dmp all custom 100 dump.lammpstrj id type q x y z

fix myspec all reaxff/species 5 1 5 species.log element Si O
\end{lcverbatim}

\noindent Then, let us use \textit{fix nvt} instead of \textit{fix npt} to apply a
thermostat but no barostat:

\begin{lcverbatim}
fix mynvt all nvt temp 300.0 300.0 100
timestep 0.5
\end{lcverbatim}

\noindent \begin{tcolorbox}[colback=mylightblue!5!white,colframe=mylightblue!75!black,title=Note]

\vspace{0.25cm} \noindent Here, no barostat is used because the box volume will be imposed by
the \textit{fix deform}.
\end{tcolorbox}

\noindent Let us run for 5000 steps without deformation, then apply the \textit{fix deform}
for elongating progressively the box along \textit{x} during 25000 steps. Add the
following line to \textit{input.lammps}:

\begin{lcverbatim}
run 5000

fix mydef all deform 1 x erate 5e-5

run 25000

write_data silica-deformed.data
\end{lcverbatim}

\noindent During the deformation, the charge values progressively evolve until the structure
eventually breaks down. After the structure breaks down, the charges
equilibrate near new average values that differ from the starting averages.
The difference between the initial and the final charges can be explained by
the presence of defects as well as new solid/vacuum interfaces, and the fact
that surface atoms typically have different charges compared to bulk atoms.

\vspace{0.25cm} \noindent [legend-to-add]Figure: Average charge per atom of the silicon (a) and oxygen (b). The
[legend-to-add]vertical dashed lines mark the beginning of the deformation.

\vspace{0.25cm} \noindent There is also a strong increase in temperature during the rupture of the
material.

\vspace{0.25cm} \noindent [legend-to-add]Figure: Temperature of the silica system over time.

\vspace{0.25cm} \noindent At the end of the deformation, one can visualize the broken material using
VMD. Notice the different charge values of the atoms located near the vacuum
interfaces, compared to the atoms located in the bulk of the material.

\vspace{0.25cm} \noindent [legend-to-add]Figure: Amorphous silicon oxide after deformation. The atoms are colored by
[legend-to-add]their charges.

\vspace{0.25cm} \noindent One can have a look at the charge distribution after deformation,
as well as during the deformation.

\vspace{0.25cm} \noindent [legend-to-add]Figure: Distribution of charge of silicon (positive, blue) and oxygen (negative, orange)
[legend-to-add]after deformation. The stars correspond to the charge distribution during deformation. 

\vspace{0.25cm} \noindent As expected, the final charge distribution slightly differs from the
previously calculated. In my case, no new species were formed during the
simulation, as can be seen from the \textit{species.log} file:

\begin{lcverbatim}
#  Timestep    No_Moles    No_Specs   Si192O384
        5           1           1           1
(...)
#  Timestep    No_Moles    No_Specs   Si192O384
    30000           1           1           1
\end{lcverbatim}

\noindent Sometimes, $\text{O}_2$ molecules are formed during the
deformation. If this is the case, the \textit{species.log} file will look like:

\begin{lcverbatim}
#  Timestep    No_Moles    No_Specs   Si192O384
          5           1           1           1
(...)
#  Timestep    No_Moles    No_Specs   Si192O382          O2
      30000           1           1           1           1
\end{lcverbatim}

\noindent \section{Decorate the surface}
Let us add hydrogen atoms to the cracked silica, and measure how the system
evolves with time. 

\vspace{0.25cm} \noindent Next to \textit{RelaxSilica/} and \textit{Deform/}, create a folder, and call it \textit{Decorate/}.
Then, let us modify the previously generated data file
\textit{silica-deformed.data} and make space for a third atom type.
Copy \textit{silica-deformed.data} from the \textit{Deform/} folder,
and modify the first lines as follow:

\begin{lcverbatim}
576 atoms
3 atom types

-12.15958814509652 32.74516585669389 xlo xhi
2.316358282925984 18.26921942866687 ylo yhi
1.3959542953413138 19.189623416252907 zlo zhi

Masses

1 28.0855
2 15.999
3 1.008

(...)
\end{lcverbatim}

\noindent Create a file named \textit{input.lammps} 
into the \textit{Decorate/} folder, and copy
the following lines into it:

\begin{lcverbatim}
units real
atom_style full

read_data silica-deformed.data
displace_atoms all move -12 0 0 # optional

pair_style reaxff NULL safezone 3.0 mincap 150
pair_coeff * * ../RelaxSilica/reaxCHOFe.ff Si O H
fix myqeq all qeq/reaxff 1 0.0 10.0 1.0e-6 reaxff maxiter 400
\end{lcverbatim}

\noindent Here, the \textit{displace$\_$atoms} command was used to move the center of the crack
near the center of the box. This step is optional but makes the visualization
of the interface in VMD easier. A different value for the shift may be
needed in your case, depending on the location of the crack.

\vspace{0.25cm} \noindent A difference with the previous input is that three atom types are specified
in the \textit{pair$\_$coeff} command, \textit{Si O H}, instead of two.

\vspace{0.25cm} \noindent Then, let us adapt some familiar commands to measure the charges of all
three types of atoms, and output the charge values into log files:

\begin{lcverbatim}
group grpSi type 1
group grpO type 2
group grpH type 3
variable qSi equal charge(grpSi)/count(grpSi)
variable qO equal charge(grpO)/count(grpO)
variable qH equal charge(grpH)/(count(grpH)+1e-10)

thermo 5
thermo_style custom step temp etotal press vol &
    v_qSi v_qO v_qH
fix myspec all reaxff/species 5 1 5 species.log element Si O H
\end{lcverbatim}

\noindent Here, the $+1\text{e}-10$ was added to the denominator of the
\textit{variable qH} in order to avoid dividing by 0 at the beginning of the
simulation.

\vspace{0.25cm} \noindent Finally, let us create a loop with 10 steps, and create two hydrogen atoms
at random locations at every step: 

\begin{lcverbatim}
fix mynvt all nvt temp 300.0 300.0 100
timestep 0.5 

label loop
variable a loop 10

variable seed equal 35672+${a} 
create_atoms 3 random 2 ${seed} NULL overlap 2.6 maxtry 50
group grpH type 3
run 2000
write_dump all custom dump.${a}.lammpstrj id type q x y z

next a
jump SELF loop

write_data decorated.data
\end{lcverbatim}

\noindent Here, a different \textit{lammpstrj} file is created for each step of the loop to
avoid creating dump files with varying numbers of atoms, which VMD can't
read.

\vspace{0.25cm} \noindent Once the simulation is over, it can be seen from the \textit{species.log} file that
all the created hydrogen atoms reacted with the $\text{SiO}_{2}$
structure to form surface groups (such as hydroxyl (-OH) groups).

\begin{lcverbatim}
# Timestep   No_Moles   No_Specs  Si192O384  H
  5          3          2         1          2
(...)
# Timestep   No_Moles No_Specs Si192O384H20
  20000      1        1        1
\end{lcverbatim}

\noindent [legend-to-add]Figure: Cracked silicon oxide after the addition of hydrogen atoms. Some
[legend-to-add]hydroxyl groups can be seen at the interfaces. The atoms are colored by
[legend-to-add]their charges.

\vspace{0.25cm} \noindent You can access the input scripts and data files that
are used in these tutorials from \href{https://github.com/lammpstutorials/lammpstutorials-inputs/}{this Github repository}.
This repository also contains the full solutions to the exercises.

\section{Going further with exercises}
\noindent Each exercise comes with a proposed solution, 
see \hyperref[solutions-label]{Solutions to the exercises}.

\subsection{Hydrate the structure}
\noindent Add water molecules to the current structure, and follow the reactions over
time.

\vspace{0.25cm} \noindent [legend-to-add]Figure: Cracked silicon oxide after the addition of water
[legend-to-add]molecules. The atoms are colored by their charges.

\subsection{A slightly acidic bulk solution}
\noindent Create a bulk water system with a few hydronium ions ($H_3O^+$
or $H^+$) using \textit{ReaxFF}. The addition of hydronium ions will make the
system acidic.

\vspace{0.25cm} \noindent [legend-to-add]Figure: Slightly acidic bulk water simulated with \textit{ReaxFF}. The atoms are
[legend-to-add]colored by their charges.


\chapter{Free energy calculation}
\label{umbrella-sampling-label}

\noindent \vspace{-1cm} \noindent \textcolor{graytitle}{\textit{{\Large Sampling a free energy barrier}}\vspace{0.5cm} }

\vspace{0.25cm} \noindent The objective of this tutorial is to measure a free
energy profile of particles across a barrier potential
using two methods; free sampling
and umbrella sampling \cite{frenkel2023understanding}.

\vspace{0.25cm} \noindent For the sake of simplicity and to reduce the computation time, the
barrier potential will be imposed artificially on the atoms.
The procedure is valid for more complex
systems, and can be adapted to many other situations, for instance 
for measuring the adsorption barrier near an interface, or for calculating
translocation barrier through a membrane.

\vspace{0.25cm} \noindent If you are completely new to LAMMPS, I recommend that
you follow this tutorial on a simple \hyperref[lennard-jones-label]{Lennard Jones fluid} first.

\begin{tcolorbox}[colback=mylightblue!5!white,colframe=mylightblue!75!black,title=What is free energy]

\vspace{0.25cm} \noindent The \textit{free energy} refers to the potential energy of a system that
is available to perform work. In molecular simulations, it is
common to calculate free energy differences between different states
or conformations of a molecular system. This can be useful in understanding
the thermodynamics of a system, predicting reaction pathways, and
determining the stability of different molecular configurations.
\end{tcolorbox}

\noindent \section{Method 1: Free sampling}
The most direct way to calculate a free energy profile is to extract
the partition function from a classic (unbiased) molecular
dynamics simulation, and then to estimate the Gibbs free
energy using 

$$\Delta G = -RT \ln(p/p_0),$$
where $\Delta G$ is the free energy difference,
$R$ the gas constant,
$T$ the temperature, 
$p$ the pressure,
and $p_0$ the reference pressure.
As an illustration, let us apply this method to an
extremely simple configuration that consists of a few
particles diffusing in a box in the presence of a position-dependent
repealing force that makes the center
of the box a relatively unfavorable area to explore.

\subsection{Basic LAMMPS parameters}
\noindent Create a folder named \textit{FreeSampling/}, and create an input script
named \textit{input.lammps} in it. Copy the following lines into it:

\begin{lcverbatim}
variable sigma equal 3.405 # Angstrom
variable epsilon equal 0.238 # Kcal/mol
variable U0 equal 1.5*${epsilon} # Kcal/mol
variable dlt equal 1.0 # Angstrom
variable x0 equal 10.0  # Angstrom

units real
atom_style atomic
pair_style lj/cut 3.822
pair_modify shift yes
boundary p p p
\end{lcverbatim}

\noindent Here, we start by defining variables for the Lennard-Jones
interaction $\sigma$ and $\epsilon$ and for
the repulsive potential $U (x)$: $U_0$, $\delta$, and $x_0$, 
see the analytical expression below.

\vspace{0.25cm} \noindent The value of 3.822 for the cut-off was chosen to 
create a WCA, purely repulsive, potential. It was calculated
as $2^{1/6} \times 3.405$ where
$3.405 = \sigma$.

\vspace{0.25cm} \noindent The system of unit \textit{real}, for which energy is in kcal/mol, distance in Ångstrom,
or time in femtosecond, has been chosen for practical reasons:
the WHAM algorithm used in the second
part of the tutorial automatically assumes the energy to
be in kcal/mol. Atoms will interact through a
Lennard-Jones potential with a cut-off equal to 
$\sigma \times 2 ^ {1/6}$ (i.e. a WCA repulsive
potential). The potential is shifted to be equal to 0 at
the cut-off using the \textit{pair$\_$modify}.

\subsection{System creation and settings}
\noindent Let us define the simulation block and randomly add atoms
by adding the following lines to \textit{input.lammps}:

\begin{lcverbatim}
region myreg block -25 25 -5 5 -25 25
create_box 1 myreg
create_atoms 1 random 60 341341 myreg overlap 1.0 maxtry 50

mass * 39.95
pair_coeff * * ${epsilon} ${sigma}
neigh_modify every 1 delay 4 check yes
\end{lcverbatim}

\noindent Here, the values for Lennard-Jones parameters $\sigma$ and
$\epsilon$ as well as the mass $m = 39.95$ grams/mole were
taken from argon.

\vspace{0.25cm} \noindent In the previous subsection, the
variables $U_0$,
$\delta$, and
$x_0$ were defined. They are used to create
the repulsive potential restricting the atoms to
explore the center of the box: 

$$U(x) = U_0 \left[ \arctan \left( \dfrac{x+x_0}{\delta} \right) - \arctan \left(\dfrac{x-x_0}{\delta} \right) \right]. $$
From the derivative of the
potential with respect to $x$, we obtain the expression
for the force that will be imposed to the atoms:

$$F(x)= \dfrac{U_0}{\delta} \left[ \dfrac{1}{(x-x_0)^2/\delta^2+1} - \dfrac{1}{(x+x_0)^2/\delta^2+1} \right].$$
The potential and force along the $x$
axis resembles:

\vspace{0.25cm} \noindent [legend-to-add]Figure: Potential $U (x)$ (a) and force $F (x)$ (b) imposed to the atoms.

\vspace{0.25cm} \noindent Let us apply energy minimization to the system,
and then impose the force $F(x)$ to all of
the atoms in the simulation using the \textit{addforce} command.
Add the following lines to \textit{input.lammps}:

\begin{lcverbatim}
minimize 1e-4 1e-6 100 1000
reset_timestep 0

variable U atom ${U0}*atan((x+${x0})/${dlt}) &
    -${U0}*atan((x-${x0})/${dlt})
variable F atom ${U0}/((x-${x0})^2/${dlt}^2+1)/${dlt} &
    -${U0}/((x+${x0})^2/${dlt}^2+1)/${dlt}
fix myadf all addforce v_F 0.0 0.0 energy v_U
\end{lcverbatim}

\noindent Finally, let us combine the \textit{fix nve} with a \textit{Langevin}
thermostat and run a molecular dynamics simulation. With
these two commands, the MD simulation is effectively in the
NVT ensemble: constant number of atoms $N$, constant
volume $V$, and constant temperature $T$. Let us
perform an equilibration of 500000 steps in total,
using a timestep of $2\,\text{fs}$
(i.e. a total duration of $1\,\text{ns}$).

\vspace{0.25cm} \noindent To make sure that $1\,\text{ns}$ is long enough, let us
record the evolution of the number of atoms in the central
(energetically unfavorable) region called \textit{mymes}:

\begin{lcverbatim}
fix mynve all nve
fix mylgv all langevin 119.8 119.8 50 1530917

region mymes block -${x0} ${x0} INF INF INF INF 
variable n_center equal count(all,mymes)
fix myat all ave/time 10 50 500 v_n_center file density_evolution.dat

timestep 2.0
thermo 10000
run 500000
\end{lcverbatim}

\subsection{Run and data acquisition}
Finally, let us record the density profile of the atoms
along the $x$ axis using the \textit{ave/chunk} command. A
total of 10 density profiles will be printed. The step count is
reset to 0 to synchronize with the output times of
\textit{fix density/number}, and the \textit{fix myat} is canceled (it has to be
canceled before a reset time):

\begin{lcverbatim}
unfix myat
reset_timestep 0

compute cc1 all chunk/atom bin/1d x 0.0 1.0
fix myac all ave/chunk 10 400000 4000000 &
    cc1 density/number file density_profile_8ns.dat
dump mydmp all atom 200000 dump.lammpstrj

thermo 100000
run 4000000
\end{lcverbatim}

\noindent This simulation with a total duration of $9\,\text{ns}$ needs a few
minutes to complete. Feel free to increase the 
duration of the last run for smoother results.

\vspace{0.25cm} \noindent [legend-to-add]Figure: Snapshot of the system. Notice that the density of atoms is lower in the central
[legend-to-add]part of the box, due to the additional force $F (x)$.

\subsection{Data analysis}
\noindent First, let us make sure that the initial equilibration of $1\,\text{ns}$
is long enough by looking at the \textit{density$\_$evolution.dat} file.

\vspace{0.25cm} \noindent [legend-to-add]Figure: Evolution of the number of atoms in the central region during equilibration. 

\vspace{0.25cm} \noindent Here, we can see that the number of atoms in the
central region, $n_\mathrm{central}$, evolves near its
equilibrium value (which is close to 0) after about $0.1\,\text{ns}$.

\vspace{0.25cm} \noindent One can also have a look at the density profile, which shows that the density in the
center of the box is about two orders of magnitude lower than inside
the reservoir.

\vspace{0.25cm} \noindent [legend-to-add]Figure: Averaged density profiles for the $8\,\text{ns}$ run. 
[legend-to-add]The value for the reference density $\rho_\text{bulk} = 0.0033$
[legend-to-add]was estimated from the raw density profiles.

\vspace{0.25cm} \noindent Then, let us plot $-R T \ln(\rho/\rho_\mathrm{bulk})$ and compare it
with the imposed (reference) potential $U$.

\vspace{0.25cm} \noindent [legend-to-add]Figure: Calculated potential $-R T \ln(\rho/\rho_\mathrm{bulk})$
[legend-to-add]compared to the imposed potential.
[legend-to-add]The calculated potential is in blue.

\vspace{0.25cm} \noindent The agreement with the expected energy profile is reasonable,
despite some noise in the central part. 

\subsection{The limits of free sampling}
\noindent If we increase the value of $U_0$, the average number of
atoms in the central region will decrease, making it
difficult to obtain a good resolution for the free energy
profile. For instance, using $U_0 = 10 \epsilon$,
not a single atom crosses the central part of the simulation,
despite the 8 ns of simulation.

\vspace{0.25cm} \noindent In that case, it is better to use the umbrella sampling method
to extract free energy profiles, see the next section.

\section{Method 2: Umbrella sampling}
\noindent Umbrella sampling is a biased molecular dynamics method,
i.e. a method in which additional forces are added to the
atoms in order to make the unfavorable states more likely
to occur \cite{frenkel2023understanding}.

\vspace{0.25cm} \noindent Several simulations (or windows) will be performed with different parameters
for the imposed biasing.

\vspace{0.25cm} \noindent Here, let us force one of the atoms to
explore the central region of the box. To do so,
let us add a potential $V$ to one
of the particle, and force it to move along the axis $x$.
The chosen path is called the axis of reaction. The results 
will be analyzed using the weighted histogram
analysis method (WHAM), which allows to remove the effect of
the bias and eventually deduce the unbiased free energy profile.

\subsection{LAMMPS input script}
\noindent Create a new folder called \textit{BiasedSampling/}, and create a new input file 
named \textit{input.lammps} in it, and copy the following lines:

\begin{lcverbatim}
variable sigma equal 3.405 # Angstrom
variable epsilon equal 0.238 # Kcal/mol
variable U0 equal 10*${epsilon} # Kcal/mol
variable dlt equal 0.5 # Angstrom
variable x0 equal 5.0  # Angstrom
variable k equal 1.5 # Kcal/mol/Angstrom^2

units real
atom_style atomic
pair_style lj/cut 3.822 # 2^(1/6) * 3.405 WCA potential
pair_modify shift yes
boundary p p p

region myreg block -25 25 -5 5 -25 25
create_box 2 myreg
create_atoms 2 single 0 0 0
create_atoms 1 random 5 341341 myreg overlap 1.0 maxtry 50

mass * 39.948
pair_coeff * * ${epsilon} ${sigma}
neigh_modify every 1 delay 4 check yes
group topull type 2

variable U atom ${U0}*atan((x+${x0})/${dlt}) &
    -${U0}*atan((x-${x0})/${dlt})
variable F atom ${U0}/((x-${x0})^2/${dlt}^2+1)/${dlt} &
    -${U0}/((x+${x0})^2/${dlt}^2+1)/${dlt}
fix pot all addforce v_F 0.0 0.0 energy v_U

fix mynve all nve
fix mylgv all langevin 119.8 119.8 50 1530917
timestep 2.0
thermo 100000
run 500000
reset_timestep 0

dump mydmp all atom 1000000 dump.lammpstrj
\end{lcverbatim}

\noindent So far, this code resembles the one of Method 1,
except for the additional particle of type 2. This
particle is identical to the particles of type 1 (same
mass and Lennard-Jones parameters), but will be exposed to the
biasing potential.

\vspace{0.25cm} \noindent Let us create a loop with 50 steps, and move progressively
the center of the bias potential by an increment of 0.1 nm.
Add the following lines into \textit{input.lammps}:

\begin{lcverbatim}
variable a loop 50
label loop
variable xdes equal ${a}-25
variable xave equal xcm(topull,x)
fix mytth topull spring tether ${k} ${xdes} 0 0 0
run 200000
fix myat1 all ave/time 10 10 100 v_xave v_xdes &
    file data-k1.5/position.${a}.dat
run 1000000
unfix myat1
next a
jump SELF loop
\end{lcverbatim}

\noindent A folder named \textit{data-k1.5/} needs to be created within \textit{BiasedSampling/}.

\vspace{0.25cm} \noindent The spring command serves to impose the
additional harmonic potential with the spring constant $k$.
Note that the value of $k$ should be chosen with care,
if $k$ is too small, the particle won't follow the biasing potential
center, if $k$ is too large, there will be no overlapping between the 
different windows. See the side note named \textit{on the choice of k} below.

\vspace{0.25cm} \noindent The center of the harmonic potential $x_\text{des}$
successively takes values from -25 to 25. For each value of
$x_\text{des}$, an equilibration step of 0.4 ns is
performed, followed by a step of 2 ns during which the
position along $x$ of the particle is saved in data
files (one data file per value of $x_\text{des}$). You
can always increase the duration of the runs for better samplings.

\subsection{WHAM algorithm}
\noindent In order to generate the free energy profile from the density distribution,
let us use the WHAM algorithm \cite{grossfieldimplementation}. 

\vspace{0.25cm} \noindent You can download it from \href{http://membrane.urmc.rochester.edu/?page_id=126}{Alan Grossfield} website, and compile it using: 

\begin{lcverbatim}
cd wham
make clean
make
\end{lcverbatim}

\noindent The compilation creates an executable called \textit{wham} that you can 
copy in the \textit{BiasedSampling/} folder. Alternatively, use 
the \href{https://lammpstutorials.github.io/lammpstutorials-inputs/level3/free-energy-calculation/BiasedSampling/wham-release-2.0.11.tgz}{version 2.0.11} I have downloaded, or try your luck with the version 
I precompiled: \href{https://lammpstutorials.github.io/lammpstutorials-inputs/level3/free-energy-calculation/BiasedSampling/wham}{precompiled wham}.

\vspace{0.25cm} \noindent In order to apply the WHAM algorithm to our simulation, we
first need to create a metadata file. This file simply
contains 

\begin{itemize}
\item the paths to all the data files,
\item the value of $x_\text{des}$,
\item and the values of $k$.
\end{itemize}

\vspace{0.25cm} \noindent To generate the \textit{metadata.txt} file, you can run this Python script
from the \textit{BiasedSampling/} folder:

\begin{lcverbatim}
import os

k=1.5 
folder='data-k1.5/'

f = open("metadata.dat", "w")
for n in range(-50,50):
    datafile=folder+'position.'+str(n)+'.dat'
    if os.path.exists(datafile):
        # read the imposed position is the expected one
        with open(datafile) as g:
            _ = g.readline()
            _ = g.readline()
            firstline = g.readline()
        imposed_position = firstline.split(' ')[-1][:-1]
        # write one file per file
        f.write(datafile+' '+str(imposed_position)+' '+str(k)+'\n')
f.close()
\end{lcverbatim}

\noindent Here, $k$ is in kcal/mol.
The generated file named \textit{metadata.dat} looks like this:

\begin{lcverbatim}
./data-k1.5/position.1.dat -24 1.5
./data-k1.5/position.2.dat -23 1.5
./data-k1.5/position.3.dat -22 1.5
(...)
./data-k1.5/position.48.dat 23 1.5
./data-k1.5/position.49.dat 24 1.5
./data-k1.5/position.50.dat 25 1.5
\end{lcverbatim}

\noindent Alternatively, you can download this \href{https://lammpstutorials.github.io/lammpstutorials-inputs/level3/free-energy-calculation/BiasedSampling/metadata.dat}{metadata.dat} file.
Then, simply run the following command in the terminal:

\begin{lcverbatim}
./wham -25 25 50 1e-8 119.8 0 metadata.dat PMF.dat
\end{lcverbatim}

\noindent where -25 and 25 are the boundaries, 50 is the number of bins,
1e-8 the tolerance, and 119.8 the temperature. A file named
PMF.dat has been created and contains the free energy
profile in Kcal/mol.

\vspace{0.25cm} \noindent Again, one can compare the result of the PMF with the imposed potential $U$,
which shows that the agreement is excellent.

\vspace{0.25cm} \noindent [legend-to-add]Figure: Calculated potential using umbrella sampling compared to
[legend-to-add]the imposed potential. The calculated potential is in blue.

\vspace{0.25cm} \noindent We can see that the agreement is quite good despite the very short calculation time
and the very high value for the energy barrier. Obtaining the same 
results with Free Sampling would require performing extremely long
and costly simulations.

\vspace{0.25cm} \noindent You can access the input scripts and data files that
are used in these tutorials from \href{https://github.com/lammpstutorials/lammpstutorials-inputs/}{this Github repository}.
This repository also contains the full solutions to the exercises.

\subsection{Side note: on the choice of k}
\noindent As already stated, one difficult part of umbrella sampling is to choose the value of $k$.
Ideally, you want the biasing potential to be strong enough to force
the chosen atom to move along the axis, and you also want the
fluctuations of the atom position to be large enough to
have some overlap in the density probability of two
neighbor positions. Here, 3 different values of $k$ are being tested.

\vspace{0.25cm} \noindent [legend-to-add]Figure: Density probability for each run with $k = 0.15\,\text{Kcal}/\text{mol}/Å^2$ (a),
[legend-to-add]$k = 1.5\,\text{Kcal}/\text{mol}/Å^2$ (b),
[legend-to-add]and $k = 15\,\text{Kcal}/\text{mol}/Å^2$ (c).

\vspace{0.25cm} \noindent If $k$ is too small, the biasing potential is too weak to
force the particle to explore the 
region of interest, making it impossible to reconstruct the PMF.

\vspace{0.25cm} \noindent If $k$ is too large, the biasing potential is too large 
compared to the potential one wants to probe, which reduces the 
sensitivity of the method.

\section{Going further with exercises}
\noindent Each exercise comes with a proposed solution, 
see \hyperref[solutions-label]{Solutions to the exercises}.

\subsection{The binary fluid that won't mix}
\noindent 1 - Create the system\textit{}

\vspace{0.25cm} \noindent Create a molecular simulation with two species of respective types 1 and 2.
Apply different potentials $U1$ and $U2$ on particles of
types 1 and 2, respectively, so that particles of type 1 are excluded
from the center of the box, while at the same time particles
of type 2 are excluded from the rest of the box.

\vspace{0.25cm} \noindent [legend-to-add]Figure: Particles of type 1 and 2 separated by two different potentials.

\vspace{0.25cm} \noindent 2 - Measure the PMFs\textit{}

\vspace{0.25cm} \noindent Using the same protocol as the one used in the tutorial
(i.e. umbrella sampling with the wham algorithm),
extract the PMF for each particle type.

\vspace{0.25cm} \noindent [legend-to-add]Figure: PMFs calculated for both atom types. 

\subsection{Particles under convection}
\noindent Use a similar simulation as the one from the tutorial,
with a repulsive potential in the center
of the box. Add force to the particles
and force them to flow in the $x$ direction.

\vspace{0.25cm} \noindent Re-measure the potential in the presence of the flow, and observe the difference
with the reference case in the absence of flow.

\vspace{0.25cm} \noindent [legend-to-add]Figure: PMF calculated in the presence of a net force that is inducing
[legend-to-add]the convection of the particles from left to right. 

\subsection{Surface adsorption of a molecule}
\noindent Apply umbrella sampling to calculate the free energy profile
of ethanol in the direction normal to a crystal solid surface
(here made of sodium chloride). Find the \href{https://lammpstutorials.github.io/lammpstutorials-inputs/level3/free-energy-calculation/Exercises/MoleculeAdsorption/init.data}{topology files}
and \href{https://lammpstutorials.github.io/lammpstutorials-inputs/level3/free-energy-calculation/Exercises/MoleculeAdsorption/PARM.lammps}{parameter file}.

\vspace{0.25cm} \noindent Use the following lines for starting the \textit{input.lammps}:

\begin{lcverbatim}
units real # style of units (A, fs, Kcal/mol)
atom_style full # molecular + charge
bond_style harmonic
angle_style harmonic
dihedral_style harmonic
boundary p p p # periodic boundary conditions
pair_style lj/cut/coul/long 10 # cut-off 1 nm
kspace_style pppm 1.0e-4
pair_modify mix arithmetic tail yes
\end{lcverbatim}

\noindent The PMF normal to a solid wall serves to indicate the free energy of adsorption,
which can be calculated from the difference between the PMF far
from the surface, and the PMF at the wall.

\vspace{0.25cm} \noindent [legend-to-add]Figure: A single ethanol molecule next to a crystal NaCl(100) wall.

\vspace{0.25cm} \noindent The PMF shows a minimum near the solid surface, which indicates a good
affinity between the wall and the molecule.

\vspace{0.25cm} \noindent [legend-to-add]Figure: PMF for a single ethanol molecule next to a NaCl
[legend-to-add]solid surface. The position of the wall is in $x=0$.
[legend-to-add]The arrow highlights the difference between the energy of the 
[legend-to-add]molecule when adsorbed to the solid surface, and
[legend-to-add]the energy far from the surface. This difference corresponds to the
[legend-to-add]free energy of adsorption.

\vspace{0.25cm} \noindent Alternatively to using ethanol, feel free to download the molecule of your choice, for 
instance from the Automated Topology Builder (ATB). Make your life simpler
by choosing a small molecule like CO2.


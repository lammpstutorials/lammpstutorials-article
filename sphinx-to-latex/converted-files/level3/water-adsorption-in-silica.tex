\chapter{Water adsorption in silica}
\label{gcmc-silica-label}

\noindent \vspace{-1cm} \noindent \textcolor{graytitle}{\textit{{\Large Dealing with a varying number of molecules}}\vspace{0.5cm} }

\vspace{0.25cm} \noindent The objective of this tutorial is to combine molecular
dynamics and grand canonical Monte Carlo simulations to
compute the adsorption of water molecules in a cracked silica material.

\vspace{0.25cm} \noindent This tutorial illustrates the use of the grand canonical
ensemble in molecular simulation, an open ensemble in which
the number of atoms within the simulation box is not constant.
When using the grand canonical ensemble, it is possible to impose
the chemical potential (or pressure, or fugacity) of a given fluid
in a nanoporous structure.

\vspace{0.25cm} \noindent If you are completely new to LAMMPS, I recommend that
you follow this tutorial on a simple \hyperref[lennard-jones-label]{Lennard Jones fluid} first.

\section{Generation of the silica block}
\noindent Let us first generate a block of amorphous silica (SiO2). To do
so, we are going to replicate a building block containing 3
Si and 6 O atoms. 

\vspace{0.25cm} \noindent Create two folders side by side, and name them respectively \textit{Potential/}
and \textit{SilicaBlock/}.

\vspace{0.25cm} \noindent An initial data file for the SiO atoms can be
downloaded by clicking \href{https://lammpstutorials.github.io/lammpstutorials-inputs/level3/water-adsorption-in-silica/SilicaBlock/SiO.data}{here}.
Save it in \textit{SilicaBlock/}. This data file
contains the coordinates of the 9 atoms, their masses, and
their charges. The \textit{.data} file can be directly read by LAMMPS using the
\textit{read$\_$data} command. Let us replicate these atoms using
LAMMPS, and apply an annealing procedure to obtain a block
of amorphous silica.

\begin{tcolorbox}[colback=mylightblue!5!white,colframe=mylightblue!75!black,title=About annealing procedure]

\vspace{0.25cm} \noindent The annealing procedure consists of adjusting the system temperature in successive steps.
Here, a large initial temperature is chosen to ensure the melting of the SiO2 structure.
Then, several steps are used to progressively cool down the system until it solidifies and forms 
amorphous silica. Depending on the material, different cooling velocities can sometimes
lead to different crystal structures or different degrees of defect.
\end{tcolorbox}

\subsection{Vashishta potential}
Create a new input file named \textit{input.lammps} in the \textit{SilicaBlock/} folder, and copy
the following lines in it:

\begin{lcverbatim}
units metal
boundary p p p
atom_style full
pair_style vashishta
neighbor 1.0 bin
neigh_modify delay 1
\end{lcverbatim}

\noindent The main difference with some of the previous tutorials is the use of 
the \textit{Vashishta} pair style. Download the \textit{Vashishta} potential by
clicking \href{https://lammpstutorials.github.io/lammpstutorials-inputs/level3/water-adsorption-in-silica/Potential/SiO.1990.vashishta}{here},
and copy it within the \textit{Potential/} folder.

\begin{tcolorbox}[colback=mylightblue!5!white,colframe=mylightblue!75!black,title=About the Vashishta potential]

\vspace{0.25cm} \noindent The \href{https://pubmed.ncbi.nlm.nih.gov/9993674/}{Vashishta}
potential is a bond-angle energy-based potential, it
deduces the bonds between atoms from their relative
positions :cite:`vashishta1990interaction`. Therefore, there is no need to
provide the bond and angle information as we do with classic force fields
like GROMOS or AMBER. When used with LAMMPS, the \textit{Vashishta}
potential requires the use of the \textit{metal} units system. 
Bond-angle energy-based potentials
are more computationally heavy than classical force
fields and require the use of a smaller timestep, but
they allow for the modeling of bond formation and
breaking, which is what we need here as we want to create
a crack in the silica.
\end{tcolorbox}

\noindent Let us then import the system made of 9 atoms, and replicate it four times in all three
directions of space, thus creating a system with 576 atoms. Add the following lines
to \textit{input.lammps}:

\begin{lcverbatim}
read_data SiO.data
replicate 4 4 4
\end{lcverbatim}

\noindent Then, let us specify the pair coefficients by indicating
that the first atom type is \textit{Si}, and
the second is \textit{O}. Let us also
add a dump command for printing out the positions of the
atoms every 5000 steps:

\begin{lcverbatim}
pair_coeff * * ../Potential/SiO.1990.vashishta Si O
\end{lcverbatim}

\noindent Let us add some commands to \textit{input.lammps} to help us follow the evolution of the system,
such as its temperature, volume, and potential energy:

\begin{lcverbatim}
dump dmp all atom 5000 dump.lammpstrj
variable myvol equal vol
variable mylx equal lx
variable myly equal ly
variable mylz equal lz
variable mypot equal pe
variable mytemp equal temp
fix myat1 all ave/time 10 100 1000 v_mytemp file temperature.dat
fix myat2 all ave/time 10 100 1000 & 
v_myvol v_mylx v_myly v_mylz file dimensions.dat
fix myat3 all ave/time 10 100 1000 v_mypot file potential-energy.dat
thermo 1000
\end{lcverbatim}

\subsection{Annealing procedure}
Finally, let us create the last part of our script. The
annealing procedure is made of four consecutive runs.
First, a $50\,\text{ps}$
phase at $T = 6000\,\text{K}$
and isotropic pressure coupling with desired pressure $p = 100\,\text{atm}$:

\begin{lcverbatim}
velocity all create 6000 4928459 rot yes dist gaussian
fix npt1 all npt temp 6000 6000 0.1 iso 100 100 1
timestep 0.001
run 50000
\end{lcverbatim}

\noindent Then, a second phase during which the system is cooled down
from $T = 6000\,\text{K}$
to $T = 4000\,\text{K}$.
An anisotropic pressure coupling is used, allowing all three
dimensions of the box to evolve independently from one another:

\begin{lcverbatim}
fix npt1 all npt temp 6000 4000 0.1 aniso 100 100 1
run 50000
\end{lcverbatim}

\noindent Then, let us cool down the system
further while also reducing the pressure, then perform a
small equilibration step at the final desired condition, $T = 300\,\text{K}$
and $p = 1\,\text{atm}$.

\begin{lcverbatim}
fix npt1 all npt temp 4000 300 0.1 aniso 100 1 1
run 200000
fix npt1 all npt temp 300 300 0.1 aniso 1 1 1
run 50000

write_data amorphousSiO.data
\end{lcverbatim}

\noindent \textit{Disclaimer --} I created this procedure by intuition and
not from proper calibration, do not copy it without
making your tests if you intend to publish your results.

\begin{tcolorbox}[colback=mylightblue!5!white,colframe=mylightblue!75!black,title=Anisotropic versus isotropic barostat]

\vspace{0.25cm} \noindent Here, an isotropic barostat is used for the melted phase at $T = 6000\,\text{K}$,
and then an anisotropic barostat is used for all following phases. With the anisotropic 
barostat, all three directions of space are adjusted independently from one another. Such
anisotropic barostat is usually a better choice for a solid phase. For a
liquid or a gas, the isotropic barostat is usually the best choice.
\end{tcolorbox}

\noindent The simulation takes about 15-20 minutes on 4 CPU cores.

\vspace{0.25cm} \noindent Let us check the evolution of the temperature from the \textit{temperature.dat} file.
Apart from an initial spike (which may be due to an initial
bad configuration, probably harmless here),
the temperature follows well the desired annealing procedure.

\vspace{0.25cm} \noindent [legend-to-add]Figure: Temperature of the system during annealing. The vertical dashed lines
[legend-to-add]mark the transition between the different phases of the simulations.

\vspace{0.25cm} \noindent Let us also make sure that the box was indeed deformed isotropically during the first 
stage of the simulation, and then anisotropically by plotting the evolution of the
box dimensions over time.

\vspace{0.25cm} \noindent [legend-to-add]Figure: Box dimensions during annealing. The vertical dashed lines
[legend-to-add]mark the transition between the different phases of the simulations.

\vspace{0.25cm} \noindent [legend-to-add]Figure: Snapshot of the final amorphous silica (SiO2) with Si atom in yellow and
[legend-to-add]O atoms in red.

\vspace{0.25cm} \noindent After running the simulation, the final LAMMPS topology file named
\textit{amorphousSiO.data} will be located in \textit{SilicaBlock/}. Alternatively, if you are only interested in the
next steps of this tutorial, you can download it by clicking
\href{https://lammpstutorials.github.io/lammpstutorials-inputs/level3/water-adsorption-in-silica/SilicaBlock/amorphousSiO.data}{here}.

\begin{tcolorbox}[colback=mylightblue!5!white,colframe=mylightblue!75!black,title=Tip for research project]

\vspace{0.25cm} \noindent In the case of a research project, the validity of the generated
structure must be tested and compared to reference values, ideally from
experiments. For instance, radial distribution functions or Young modulus
can both be compared to experimental values. This is beyond the
scope of this tutorial.
\end{tcolorbox}

\noindent \section{Cracking the silica}
Let us dilate the block of silica until a
crack forms. Create a new folder called \textit{Cracking/} next to \textit{SilicaBlock/},
as well as a new \textit{input.lammps} file starting with familiar lines as
previously:

\begin{lcverbatim}
units metal
boundary p p p
atom_style full
neighbor 1.0 bin
neigh_modify delay 1

read_data ../SilicaBlock/amorphousSiO.data

pair_style vashishta
pair_coeff * * ../Potential/SiO.1990.vashishta Si O
dump dmp all atom 1000 dump.lammpstrj
\end{lcverbatim}

\noindent Let us progressively increase the size of the
box in the x direction, thus forcing the silica to deform
and eventually crack. To do
so, a loop based on the jump command is used. At
every step of the loop, the box dimension over x will
be multiplied by a scaling factor 1.005. Add the following lines to
the \textit{input.lammps}:

\begin{lcverbatim}
fix nvt1 all nvt temp 300 300 0.1
timestep 0.001
thermo 1000
variable var loop 45
label loop
change_box all x scale 1.005 remap
run 2000
next var
jump input.lammps loop
run 20000
write_data dilatedSiO.data
\end{lcverbatim}

\noindent The \textit{fix nvt} is used
to control the temperature of the system, while the \textit{change$\_$box} command
imposes incremental deformations of the box.
Different scaling factors or/and different numbers of 
steps can be used to generate different defects in the silica.

\begin{tcolorbox}[colback=mylightblue!5!white,colframe=mylightblue!75!black,title=On using barostat during deformation]

\vspace{0.25cm} \noindent Here, box deformations are applied in the x direction, while the 
y and z box dimensions are kept constants. 

\vspace{0.25cm} \noindent Another possible choice is to apply a barostat along the y and z 
directions, allowing for the system to adjust to the stress. In LAMMPS, 
this can be done by using :

\begin{lcverbatim}
fix npt1 all npt temp 300 300 0.1 y 1 1 1 z 1 1 1
\end{lcverbatim}

\noindent instead of:

\begin{lcverbatim}
fix nvt1 all nvt temp 300 300 0.1
\end{lcverbatim}

\noindent \end{tcolorbox}

\vspace{0.25cm} \noindent [legend-to-add]Figure: Block of silica after deformation, with some visible holes.

\vspace{0.25cm} \noindent After the dilatation, a final equilibration step of duration 20
picoseconds is performed. If you look at the \textit{dump.lammpstrj} file
using VMD, you can see the dilatation occurring step-by-step, and the
atoms progressively adjusting to the box dimensions. 

\vspace{0.25cm} \noindent At first, the deformations
are reversible (elastic regime). At some point, bonds
start breaking and dislocations appear (plastic regime). 

\vspace{0.25cm} \noindent Alternatively, you can download the final state directly by clicking
\href{https://lammpstutorials.github.io/lammpstutorials-inputs/level3/water-adsorption-in-silica/Cracking/dilatedSiO.data}{here}.

\begin{tcolorbox}[colback=mylightblue!5!white,colframe=mylightblue!75!black,title=Passivated silica]

\vspace{0.25cm} \noindent In ambient conditions, some of the surface SiO2 atoms are chemically
passivated by forming covalent bonds with hydrogen (H)
atoms. For the sake of simplicity, we are not going to
add surface hydrogen atoms here. An example of a procedure allowing
for properly inserting hydrogen atoms is used
in \hyperref[reactive-silicon-dioxide-label]{Reactive silicon dioxide}.
\end{tcolorbox}

\noindent \section{Adding water}
In order to add the water molecules to the silica, we are
going to use the Monte Carlo method in the grand canonical
ensemble (GCMC). In short, the system is put into contact
with a virtual reservoir of a given chemical potential
$\mu$, and multiple attempts to insert water
molecules at random positions are made. Each attempt is
either accepted or rejected based on energy considerations. Find more details
in classical textbooks \cite{frenkel2023understanding}.

\subsection{Using hydrid potentials}
\noindent Create a new folder called \textit{Addingwater/}. Download and save the
\href{https://lammpstutorials.github.io/lammpstutorials-inputs/level3/water-adsorption-in-silica/AddingWater/H2O.mol}{template} file for the
water molecule within \textit{Addingwater/}.

\vspace{0.25cm} \noindent Create a new input file called \textit{input.lammps}
within \textit{Addingwater/}, and copy the
following lines into it:

\begin{lcverbatim}
units metal
boundary p p p
atom_style full
neighbor 1.0 bin
neigh_modify delay 1
pair_style hybrid/overlay vashishta lj/cut/tip4p/long 3 4 1 1 0.1546 10
kspace_style pppm/tip4p 1.0e-4
bond_style harmonic
angle_style harmonic
\end{lcverbatim}

\noindent There are several differences with the previous input files
used in this tutorial. From now on, the system will combine water and silica,
and therefore two force fields are combined: Vashishta for
SiO, and lj/cut/tip4p/long for TIP4P water model (here 
the TIP4P/2005 model is used \cite{abascal2005general}).
Combining the two force fields is done using the \textit{hybrid/overlay} pair style.

\begin{tcolorbox}[colback=mylightblue!5!white,colframe=mylightblue!75!black,title=About hybrid and hybrid/overlay pair style]

\vspace{0.25cm} \noindent From the LAMMPS documentation:
The hybrid and hybrid/overlay styles enable the use
of multiple pair styles in one simulation. With the hybrid style,
exactly one pair style is assigned to each pair of atom types.
With the hybrid/overlay and hybrid/scaled styles, one or more pair
styles can be assigned to each pair of atom types.
\end{tcolorbox}

\noindent The \textit{kspace} solver is used to calculate the long
range Coulomb interactions associated with \textit{tip4p/long}.
Finally, the style for the bonds and angles
of the water molecules are defined, although they are not important
since it is a rigid water model.

\vspace{0.25cm} \noindent Before going further, we also need to make a few changes to our data file.
Currently, \textit{dilatedSiO.data} only includes two atom types, but
we need four. Copy the previously generated \textit{dilatedSiO.data}
file within \textit{Addingwater/}. Currently, \textit{dilatedSiO.data} starts with:

\begin{lcverbatim}
576 atoms
2 atom types

-5.512084438507452 26.09766215010596 xlo xhi
-0.12771230207837192 20.71329001367807 ylo yhi
3.211752393088563 17.373825318513106 zlo zhi

Masses

1 28.0855
2 15.9994

Atoms # full

(...)
\end{lcverbatim}

\noindent Make the following changes to allow for the addition of water
molecules. Modify the file so that it looks like the following 
(with 4 atom types, 1 bond type, 1 angle type, and four masses):

\begin{lcverbatim}
576 atoms
4 atom types
1 bond types
1 angle types

2 extra bond per atom
1 extra angle per atom
2 extra special per atom

0.910777522101565 19.67480018949893 xlo xhi
2.1092682236518137 18.476309487947546 ylo yhi
-4.1701120819606885 24.75568979356097 zlo zhi

Masses

1 28.0855
2 15.9994
3 15.9994
4 1.008

Atoms # full

(...)
\end{lcverbatim}

\noindent Doing so, we anticipate that there will be 4 atom types in
the simulations, with O and H of H2O having indexes 3 and 4,
respectively. There will also be 1 bond type and 1 angle
type. The extra bond, extra angle, and extra special lines
are here for memory allocation. 

\vspace{0.25cm} \noindent We can continue to fill in the
\textit{input.lammps} file, by adding the system definition:

\begin{lcverbatim}
read_data dilatedSiO.data
molecule h2omol H2O.mol
lattice sc 3
create_atoms 0 box mol h2omol 45585
lattice none 1

group SiO type 1 2
group H2O type 3 4
\end{lcverbatim}

\noindent After reading the data file and defining the h2omol molecule
from the txt file, the \textit{create$\_$atoms} command is used to
include some water molecules in the system on a 
simple cubic lattice. Not adding a molecule before starting the
GCMC steps usually lead to failure. Note that here,
most water molecules overlap with the silica. These 
overlapping water molecules will be deleted before 
starting the simulation. 

\vspace{0.25cm} \noindent Then, add the following settings to \textit{input.lammps}:

\begin{lcverbatim}
pair_coeff * * vashishta ../Potential/SiO.1990.vashishta Si O NULL NULL
pair_coeff * * lj/cut/tip4p/long 0 0
# epsilonSi = 0.00403, sigmaSi = 3.69
# epsilonO = 0.0023, sigmaO = 3.091
pair_coeff 1 3 lj/cut/tip4p/long 0.0057 4.42
pair_coeff 2 3 lj/cut/tip4p/long 0.0043 3.12
pair_coeff 3 3 lj/cut/tip4p/long 0.008 3.1589
pair_coeff 4 4 lj/cut/tip4p/long 0.0 0.0
bond_coeff 1 0 0.9572
angle_coeff 1 0 104.52

variable oxygen atom "type==3"
group oxygen dynamic all var oxygen
variable nO equal count(oxygen)
fix myat1 all ave/time 100 10 1000 v_nO file numbermolecule.dat

fix shak H2O shake 1.0e-4 200 0 b 1 a 1 mol h2omol
\end{lcverbatim}

\noindent The force field Vashishta applies only to Si (type 1)
and O of SiO2 (type 2),
and not to the O and H of H2O, thanks to the NULL
parameters used for atoms of types 3 and 4. 

\vspace{0.25cm} \noindent Pair coefficients for lj/cut/tip4p/long are
defined between O atoms, as well as between
O(SiO)-O(H2O) and Si(SiO)-O(H2O). Therefore, the fluid-solid 
interactions will be set by Lennard-Jones and Coulomb potentials. 

\vspace{0.25cm} \noindent The number of oxygen atoms from water molecules (i.e. the number of molecules)
will be printed in the file \textit{numbermolecule.dat}.

\vspace{0.25cm} \noindent The shake algorithm is used to
maintain the shape of the water molecules over time. Some of
these features have been seen in previous tutorials.

\vspace{0.25cm} \noindent Let us delete the overlapping water molecules, and print the
positions of the remaining atoms in a \textit{.lammpstrj} file by adding the following
lines to \textit{input.lammps}:

\begin{lcverbatim}
delete_atoms overlap 2 H2O SiO mol yes
dump dmp all atom 1000 dump.init.lammpstrj
\end{lcverbatim}

\subsection{GCMC simulation}
To prepare for the GCMC simulation,
let us make the first equilibration step
by adding the following lines to \textit{input.lammps}:

\begin{lcverbatim}
compute_modify thermo_temp dynamic yes
compute ctH2O H2O temp
compute_modify ctH2O dynamic yes
fix mynvt1 H2O nvt temp 300 300 0.1
fix_modify mynvt1 temp ctH2O
compute ctSiO SiO temp
fix mynvt2 SiO nvt temp 300 300 0.1
fix_modify mynvt2 temp ctSiO
timestep 0.001
thermo 1000
run 5000
\end{lcverbatim}

\noindent \begin{tcolorbox}[colback=mylightblue!5!white,colframe=mylightblue!75!black,title=On thermostating groups instead of the entire system]

\vspace{0.25cm} \noindent Two different thermostats are used for SiO and for H2O, respectively. Using 
separate thermostats are usually better when the system contains two separate
species, such as a solid and a
liquid. It is particularly important to use two thermostats
here because the number of water molecules will fluctuate with time.
\end{tcolorbox}

\noindent The \textit{compute$\_$modify} with 
\textit{dynamic yes} for water is used to specify that the
number of molecules is not constant.

\vspace{0.25cm} \noindent Finally, let us use the \textit{fix gcmc} and perform the grand
canonical Monte Carlo steps. Add the following lines into \textit{input.lammps}:

\begin{lcverbatim}
variable tfac equal 5.0/3.0
variable xlo equal xlo+0.1
variable xhi equal xhi-0.1
variable ylo equal ylo+0.1
variable yhi equal yhi-0.1
variable zlo equal zlo+0.1
variable zhi equal zhi-0.1
region system block ${xlo} ${xhi} ${ylo} ${yhi} ${zlo} ${zhi} 
fix fgcmc H2O gcmc 100 100 0 0 65899 300 -0.5 0.1 &
    mol h2omol tfac_insert ${tfac} group H2O shake shak &
    full_energy pressure 10000 region system
run 45000
write_data SiOwithwater.data
write_dump all atom dump.lammpstrj
\end{lcverbatim}

\noindent \begin{tcolorbox}[colback=mylightblue!5!white,colframe=mylightblue!75!black,title=Dirty fix]

\vspace{0.25cm} \noindent The region \textit{system} was created to avoid the error Fix gcmc
region extends outside simulation box
which seems to occur with the 2Aug2023 LAMMPS version.
\end{tcolorbox}

\noindent The \textit{tfac$\_$insert} option ensures that the correct estimate is
made for the temperature of the inserted water molecules by
taking into account the internal degrees of freedom. Running
this simulation, you should see the number of molecules
increasing progressively. When using the pressure argument,
LAMMPS ignores the value of the chemical potential [here $\mu = -0.5\,\text{eV}$
which corresponds roughly to ambient conditions (i.e. $\text{RH} \approx 50\,\%$).]
The large pressure value of 10000 bars was chosen to ensure that 
some successful insertions of molecules would occur during the 
extremely short duration of this simulation.

\vspace{0.25cm} \noindent When you run the simulation, make sure that some water molecules 
remain in the system after the \textit{delete$\_$atoms} command. You can control 
that either using the log file or using the \textit{numbermolecule.dat} data file.

\vspace{0.25cm} \noindent You can see, by looking at the log file, that 280 molecules
were added by the \textit{create$\_$atoms} command (the exact number you get may differ):

\begin{lcverbatim}
Created 840 atoms
\end{lcverbatim}

\noindent You can also see that 258 molecules were immediately deleted,
leaving 24 water molecules (the exact number you get may differ):

\begin{lcverbatim}
Deleted 774 atoms, new total = 642
Deleted 516 bonds, new total = 44
Deleted 258 angles, new total = 22
\end{lcverbatim}

\noindent After just a few GCMC steps,
the number of molecules starts increasing.
Once the crack is filled with water molecules, the number of
molecules reaches a plateau.

\vspace{0.25cm} \noindent [legend-to-add]Figure: Number of molecules as a function of time. The dashed vertical line
[legend-to-add]marks the beginning of the GCMC step.

\vspace{0.25cm} \noindent The final number of molecules depends on the imposed pressure, 
temperature, and on the interaction between water and silica (i.e. its hydrophilicity). 

\vspace{0.25cm} \noindent [legend-to-add]Figure: Snapshot of the silica system after the adsorption of the water molecules,
[legend-to-add]with the oxygen of the water molecules represented in cyan.

\vspace{0.25cm} \noindent Note that GCMC simulations of such dense phases are usually slow to converge due to the
very low probability of successfully inserting a molecule. Here, the short simulation 
duration was made possible by the use of a large pressure.

\begin{tcolorbox}[colback=mylightblue!5!white,colframe=mylightblue!75!black,title=Vizualising varying number of molecules]

\vspace{0.25cm} \noindent By default, VMD fails to properly render systems with varying numbers of atoms.
\end{tcolorbox}

\noindent You can access the input scripts and data files that
are used in these tutorials from \href{https://github.com/lammpstutorials/lammpstutorials-inputs/}{this Github repository}.
This repository also contains the full solutions to the exercises.

\section{Going further with exercises}
\noindent Each exercise comes with a proposed solution, 
see \hyperref[solutions-label]{Solutions to the exercises}.

\subsection{Mixture adsorption}
\noindent Adapt the existing script and insert both $\text{CO}_2$ molecules
and water molecules within the silica crack using GCMC. 
Download the \href{https://lammpstutorials.github.io/lammpstutorials-inputs/level3/water-adsorption-in-silica/Exercises/MixtureH2OCO2/CO2.mol}{CO2 template}. The parameters for the
$\text{CO}_2$
molecule are the following:

\begin{lcverbatim}
pair_coeff 5 5 lj/cut/tip4p/long 0.0179 2.625854
pair_coeff 6 6 lj/cut/tip4p/long 0.0106 2.8114421 
bond_coeff 2 46.121 1.17
angle_coeff 2 2.0918 180
\end{lcverbatim}

\noindent The atom of type 5 is an oxygen of 
mass 15.9994, and the atom of type 6 is a carbon of mass 12.011.

\vspace{0.25cm} \noindent [legend-to-add]Figure: Cracked silica with adsorbed water and $\text{CO}_2$ molecules (in green).

\subsection{Adsorb water in ZIF-8 nanopores}

\noindent Use the same protocol as the one implemented in this tutorial to add water
molecules to a Zif-8 nanoporous material. A snapshot of the system with a 
few water molecules is shown on the right.

\vspace{0.25cm} \noindent Download the initial Zif-8 \href{https://lammpstutorials.github.io/lammpstutorials-inputs/level3/water-adsorption-in-silica/Exercises/Zif-8/zif-8.data}{structure},
the \href{https://lammpstutorials.github.io/lammpstutorials-inputs/level3/water-adsorption-in-silica/Exercises/Zif-8/parm.lammps}{parameters} file, and this
new \href{https://lammpstutorials.github.io/lammpstutorials-inputs/level3/water-adsorption-in-silica/Exercises/Zif-8/water.mol}{water template}. The ZIF-8 structure is made
of 7 atom types (C1, C2, C3, H2, H3, N, Zn), connected
by bonds, angles, dihedrals, and impropers. It uses the
same \textit{pair$\_$style} as water,
so there is no need to use \textit{hybrid pair$\_$style}.
Your \textit{input} file should start like this:

\begin{lcverbatim}
units real
atom_style full
boundary p p p
bond_style harmonic
angle_style harmonic
dihedral_style charmm
improper_style harmonic

pair_style lj/cut/tip4p/long 1 2 1 1 0.105 14.0
kspace_style pppm/tip4p 1.0e-5

special_bonds lj 0.0 0.0 0.5 coul 0.0 0.0 0.833
\end{lcverbatim}

\noindent An important note: here, water occupies the atom types 1 and 2,
instead of 3 and 4 in the case of SiO2 from the main section 
of the tutorial.


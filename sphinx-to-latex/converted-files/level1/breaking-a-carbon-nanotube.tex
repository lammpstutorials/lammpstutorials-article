\chapter{Pulling on a carbon nanotube}
\label{carbon-nanotube-label}

\noindent \vspace{-1cm} \noindent \textcolor{graytitle}{\textit{{\Large Stretching a carbon nanotube until it breaks}}\vspace{0.5cm} }

\vspace{0.25cm} \noindent The objective of this tutorial is to impose the deformation
of a carbon nanotube (CNT) using LAMMPS.

\vspace{0.25cm} \noindent In this tutorial, a small carbon nanotube (CNT) is simulated
within an empty box using LAMMPS. An external 
force is imposed on the CNT, and its deformation is measured over time.

\vspace{0.25cm} \noindent To illustrate the difference between classical and reactive force fields,
this tutorial is divided into two parts. Within the first part, a classical
force field is used and the bonds between the atoms of the CNT are
unbreakable. Within the second part, a reactive force field
(named AIREBO \cite{stuart2000reactive}) is used, allowing for the breaking
of chemical bonds when the CNT undergoes strong deformation.

\vspace{0.25cm} \noindent If you are completely new to LAMMPS, I recommend that
you follow this tutorial on a simple \hyperref[lennard-jones-label]{Lennard Jones fluid} first.

\section{Unbreakable bonds}
\noindent With most classical molecular dynamics force fields, the chemical bonds
between the atoms are set at the start of the simulation. Regardless of the 
forces applied to the atoms during the simulations, the bonds remain intact.
The bonds between neighbor atoms typically consist of springs with
given equilibrium distances $r_0$ and a constant $k_b$:
$U_b = k_b \left( r - r_0 \right)^2$.
Additionally, angular and dihedral constraints are usually applied to maintain
the relative orientations of neighbor atoms. 

\subsection{Create topology with VMD}
\noindent The first part of this tutorial is dedicated to creating
the initial topology with VMD. You can skip this part by
downloading directly the CNT topology \href{https://lammpstutorials.github.io/lammpstutorials-inputs/level1/breaking-a-carbon-nanotube/unbreakable-bonds/cnt_molecular.data}{here}, 
and continue with the LAMMPS part of the tutorial.

\begin{tcolorbox}[colback=mylightblue!5!white,colframe=mylightblue!75!black,title=Why use a preprocessing tool?]

\vspace{0.25cm} \noindent When the system has a complex topology, like is the case of a CNT, 
it is better to use an external preprocessing tool to create it as it would be
difficult (yet not impossible) to place the atoms in their correct position
using only LAMMPS commands. Many preprocessing tools exist, see
this \href{https://www.lammps.org/prepost.html}{non-exhaustive list} on the LAMMPS website.
\end{tcolorbox}

\noindent Open VMD, go to Extensions, Modeling, and then Nanotube Builder.
A window named Carbon Nanostructures opens up, allowing us to choose
between generating a sheet or a nanotube, made either of graphene or
of Boron Nitride (BN). For this tutorial, let us generate a carbon nanotube.
Keep all default values, and click on \textit{Generate Nanotube}.

\vspace{0.25cm} \noindent At this point, this is not a molecular dynamics simulation,
but a cloud of unconnected dots. In the VMD terminal, set the
box dimensions by typing the following commands in the VMD terminal:

\begin{lcverbatim}
molinfo top set a 80  
molinfo top set b 80            
molinfo top set c 80 
\end{lcverbatim}

\noindent The values of 80 in each direction have been chosen
so that the box is much larger than the carbon nanotube.

\vspace{0.25cm} \noindent To generate the initial LAMMPS data file, let us use \textit{Topotools}:
to generate the LAMMPS data file, enter the following command
in the VMD terminal:

\begin{lcverbatim}
topo writelammpsdata cnt_molecular.data molecular
\end{lcverbatim}

\noindent Here \textit{molecular} refers to the LAMMPS \textit{atom$\_$style}, and
\textit{cnt$\_$molecular.data} is the name of the file. 

\begin{tcolorbox}[colback=mylightblue!5!white,colframe=mylightblue!75!black,title=About TopoTools]

\vspace{0.25cm} \noindent \textit{Topotools} deduces the location of bonds, angles,
dihedrals, and impropers from the respective positions of the atoms,
and generates a \textit{.data} file that can be read by LAMMPS :cite:`kohlmeyer2017topotools`.
More details about \textit{Topotools} can be found on the
personal page of \href{https://sites.google.com/site/akohlmey/software/topotools}{Axel Kohlmeyer}.
\end{tcolorbox}

\noindent The parameters of the constraints (bond length,
dihedral coefficients, etc.) will be given later.
A new file named \textit{cnt$\_$molecular.data} has been created, it starts
like that:

\begin{lcverbatim}
700 atoms
1035 bonds
2040 angles
4030 dihedrals
670 impropers
1 atom types
1 bond types
1 angle types
1 dihedral types
1 improper types
-40.000000 40.000000  xlo xhi
-40.000000 40.000000  ylo yhi
-12.130411 67.869589  zlo zhi
(...)
\end{lcverbatim}

\noindent The \textit{cnt$\_$molecular.data} file contains information
about the positions of the carbon atoms, as well as the
identity of the atoms that are linked by \textit{bonds}, \textit{angles}, \textit{dihedrals},
and \textit{impropers} constraints.

\vspace{0.25cm} \noindent Save the \textit{cnt$\_$molecular.data} file in a folder named \textit{unbreakable-bonds/}.

\subsection{The LAMMPS input}
\noindent Create a new text file within \textit{unbreakable-bonds/} and name
it \textit{input.lammps}. Copy the following lines in it:

\begin{lcverbatim}
variable T equal 300

units real
atom_style molecular
boundary f f f
pair_style lj/cut 14

bond_style harmonic
angle_style harmonic
dihedral_style opls
improper_style harmonic

special_bonds lj 0.0 0.0 0.5

read_data cnt_molecular.data
\end{lcverbatim}

\noindent The chosen unit system is \textit{real} (therefore distances are in Ångstrom, time in femtosecond),
the \textit{atom$\_$style} is molecular (therefore atoms are dots that can be bonded with each other),
and the boundary conditions are fixed. The boundary conditions
do not matter here, as the box boundaries were placed far from the CNT. 

\vspace{0.25cm} \noindent Just like in \hyperref[lennard-jones-label]{Lennard Jones fluid},
the pair style is \textit{lj/cut} (i.e. a Lennard-Jones potential 
with a short-range cutoff) with
parameter 14, which means that only the atoms closer than 14
Ångstroms from each other interact through a Lennard-Jones
potential.

\vspace{0.25cm} \noindent The \textit{bond$\_$style}, \textit{angle$\_$style}, \textit{dihedral$\_$style}, and \textit{improper$\_$style} commands specify the
different potentials used to restrain the relative positions of the
atoms. For more details about the potentials used here, you can have a look
at the LAMMPS website, see for example
the page of the \href{https://lammps.sandia.gov/doc/dihedral_opls.html}{OPLS dihedral style} \cite{jorgensen1988opls}.

\vspace{0.25cm} \noindent The last command, \textit{read$\_$data}, imports the \textit{cnt$\_$molecular.data} file
previously generated with VMD, which contains the
information about the box size, atom positions, etc.

\begin{tcolorbox}[colback=mylightblue!5!white,colframe=mylightblue!75!black,title=About interaction between neighbors atoms]

\end{tcolorbox}

\noindent We need to specify the parameters of both bonded and
non-bonded potentials. Here, the parameters are taken from the OPLS-AA
(Optimised Potentials for Liquid Simulations-All-Atom) force 
field \cite{jorgensenDevelopmentTestingOPLS1996}.
Create a new text file in the \textit{unbreakable-bonds/}
folder and name it \textit{parm.lammps}. Copy the following lines in it:

\begin{lcverbatim}
pair_coeff 1 1 0.066 3.4
bond_coeff 1 469 1.4
angle_coeff 1 63 120
dihedral_coeff 1 0 7.25 0 0
improper_coeff 1 5 180
\end{lcverbatim}

\noindent The \textit{pair$\_$coeff} command sets the parameters for the non-bonded Lennard-Jones
interaction $\epsilon_{11} = 0.066 \, \text{kcal/mol}$
and $\sigma_{11} = 3.4 \, \text{Å}$ for the only type of atom of the
simulation; the carbon atom of type 1. 

\vspace{0.25cm} \noindent The \textit{bond$\_$coeff} provides the equilibrium distance $r_0= 1.4 \, \text{Å}$ as
well as the spring constant $k_b = 469 \, \text{kcal/mol/Å}^2$ for the harmonic
potential imposed between two neighboring carbon atoms,
where the potential is $U_b = k_b ( r - r_0)^2$. The
\textit{angle$\_$coeff} gives the equilibrium angle $\theta_0$ and
constant for the potential between three neighbor atoms :
$U_\theta = k_\theta ( \theta - \theta_0)^2$. The \textit{dihedral$\_$coeff}
and \textit{improper$\_$coeff} gives the potential for the constraints
between 4 atoms. 

\vspace{0.25cm} \noindent The file \textit{parm.lammps} is included in the
simulation by adding the following line to the \textit{input.lammps} file:

\begin{lcverbatim}
include parm.lammps
\end{lcverbatim}

\subsection{Prepare the initial state}
Before starting the molecular dynamics simulation,
let us make sure that we start from a clean initial state
by recentering the CNT at the origin (0, 0, 0). In addition, 
let us make sure that the box boundaries 
are symmetric with respect to (0, 0, 0), which is not initially the case,
as seen in \textit{cnt$\_$molecular.data}:

\begin{lcverbatim}
-40.000000 40.000000  xlo xhi
-40.000000 40.000000  ylo yhi
-12.130411 67.869589  zlo zhi
\end{lcverbatim}

\noindent Let us recenter the CNT by adding the following lines
to \textit{input.lammps}:

\begin{lcverbatim}
group carbon_atoms type 1
variable carbon_xcm equal -1*xcm(carbon_atoms,x)
variable carbon_ycm equal -1*xcm(carbon_atoms,y)
variable carbon_zcm equal -1*xcm(carbon_atoms,z)
displace_atoms carbon_atoms &
    move ${carbon_xcm} ${carbon_ycm} ${carbon_zcm}
\end{lcverbatim}

\noindent The first command includes all the atoms of type 1
(i.e. all the atoms here) in a group named \textit{carbon$\_$atoms}. 
The 3 variables, \textit{carbon$\_$xcm}, \textit{carbon$\_$ycm}, and \textit{carbon$\_$zcm} 
are used to measure
the current position of the group \textit{carbon$\_$atoms}
along all 3 directions, respectively. Then, the \textit{displace$\_$atoms} 
command move the group \textit{carbon$\_$atoms}, ensuring that its center of mass 
is located at the origin (0, 0, 0).

\vspace{0.25cm} \noindent Let us also change the box boundaries by adding the following line to \textit{input.lammps}:

\begin{lcverbatim}
change_box all x final -40 40 y final -40 40 z final -40 40
\end{lcverbatim}

\noindent \begin{tcolorbox}[colback=mylightblue!5!white,colframe=mylightblue!75!black,title=Note]

\vspace{0.25cm} \noindent Such a cleaner and more symmetrical initial state can simplify
future data analysis, but won't make any difference to 
the molecular dynamics.
\end{tcolorbox}

\noindent A displacement will be imposed on the edges of the CNT. To do so, let us isolate the
atoms from the two edges and place them into groups named \textit{rtop}
and \textit{rbot}, respectively.
Add the following lines to \textit{input.lammps}:

\begin{lcverbatim}
variable zmax equal bound(carbon_atoms,zmax)-0.5
variable zmin equal bound(carbon_atoms,zmin)+0.5
region rtop block INF INF INF INF ${zmax} INF
region rbot block INF INF INF INF INF ${zmin}
region rmid block INF INF INF INF ${zmin} ${zmax}
\end{lcverbatim}

\noindent The variable $z_\mathrm{max}$ corresponds to
the coordinate of the last atoms along $z$ minus 0.5
Ångstroms, and $z_\mathrm{min}$ to the coordinate of
the first atoms along $z$ plus 0.5 Ångstroms. Then, three
regions are defined, and correspond respectively to: $z < z_\mathrm{min}$,
(\textit{rbot}, for region bottom)
$z_\mathrm{min} > z > z_\mathrm{max}$
(\textit{rmid}, for region middle), and  
$z > z_\mathrm{max}$
(\textit{rtop}, for region top).

\vspace{0.25cm} \noindent Finally, let us define 3 groups of atoms
corresponding to the atoms located in each of the three regions,
respectively, by adding to \textit{input.lammps}:

\begin{lcverbatim}
group carbon_top region rtop
group carbon_bot region rbot
group carbon_mid region rmid
\end{lcverbatim}

\noindent The atoms of the edges as selected within the \textit{carbon$\_$top}
and \textit{carbon$\_$bot} groups can be represented with a different color.

\vspace{0.25cm} \noindent [legend-to-add]Figure: CNT with atoms from the \textit{carbon$\_$top}
[legend-to-add]and the \textit{carbon$\_$bot} groups are represented with a different color.

\vspace{0.25cm} \noindent When running a simulation, the number of atoms in each group is printed in
the terminal (and in the \textit{log.lammps} file). Always make sure that the number
of atoms in each group corresponds to what is expected, just like here:

\begin{lcverbatim}
10 atoms in group carbon_top
10 atoms in group carbon_bot
680 atoms in group carbon_mid
\end{lcverbatim}

\noindent Finally, to start from a less ideal state and create a system with some defects, 
let us randomly delete some of the carbon atoms.
In order to avoid deleting atoms that are too close to the edges,
let us define a new region name \textit{rdel} that
starts $2\,Å$
from the CNT edges.

\begin{lcverbatim}
variable zmax_del equal ${zmax}-2
variable zmin_del equal ${zmin}+2
region rdel block INF INF INF INF ${zmin_del} ${zmax_del}
group rdel region rdel
delete_atoms random fraction 0.02 no rdel NULL 482793 bond yes
\end{lcverbatim}

\noindent The \textit{delete$\_$atoms} command randomly
deletes $2\,\%$ of the atoms
from the \textit{rdel} group (i.e. about 10 atoms).

\vspace{0.25cm} \noindent [legend-to-add]Figure: CNT with \textit{10} randomly deleted atoms. 

\subsection{The molecular dynamics}
\noindent Let us specify the thermalization and the dynamics of the
system. Add the following lines to \textit{input.lammps}:

\begin{lcverbatim}
reset_atoms id sort yes
velocity carbon_mid create ${T} 48455 mom yes rot yes
fix mynve all nve
compute Tmid carbon_mid temp
fix myber carbon_mid temp/berendsen ${T} ${T} 100
fix_modify myber temp Tmid
\end{lcverbatim}

\noindent Re-setting the atom ids is necessary before using the \textit{velocity} command,
this is done by the \textit{reset$\_$atoms} command.

\vspace{0.25cm} \noindent The \textit{velocity} command gives initial velocities to
the atoms of the middle group \textit{carbon$\_$mid}, ensuring an initial temperature
of 300 K for these atoms with no overall translational momentum, \textit{mom yes},
nor rotational momentum, \textit{rot yes}.

\vspace{0.25cm} \noindent The \textit{fix nve} is applied to all atoms so that all
atom positions are recalculated at every step, and
a \textit{Berendsen} thermostat is applied to the atoms
of the group \textit{carbon$\_$mid} only \cite{berendsen1984molecular}.
The \textit{fix$\_$modify myber} ensures that the
\textit{fix Berendsen} uses the temperature of the group \textit{carbon$\_$mid} as an
input, instead of the temperature of the whole system. This is necessary
to make sure that the frozen edges won't bias the temperature. Note that the atoms
of the edges do not need a thermostat because their motion will
be restrained, see below.

\begin{tcolorbox}[colback=mylightblue!5!white,colframe=mylightblue!75!black,title=Deal with semi-frozen system]

\vspace{0.25cm} \noindent Always be careful when part of a system is frozen,
as is the case here. When some atoms are frozen, the total temperature
of the system is effectively lower than the applied temperature
because the frozen atoms have no thermal motion (their temperature
is therefore $0\,\text{K}$). 
\end{tcolorbox}

\noindent To restrain the motion of the atoms at the edges, let us add the
following commands to \textit{input.lammps}:

\begin{lcverbatim}
fix mysf1 carbon_top setforce 0 0 0
fix mysf2 carbon_bot setforce 0 0 0
velocity carbon_top set 0 0 0
velocity carbon_bot set 0 0 0
\end{lcverbatim}

\noindent The two \textit{setforce} commands cancel the forces applied on the
atoms of the two edges, respectively. The cancellation of the forces
is done at every step, and along all 3 directions of space, $x$, $y$,
and $z$, due to the use of \textit{0 0 0}. The two \textit{velocity} commands
set the initial velocities along $x$,
$y$, and $z$ to 0 for the atoms of \textit{carbon$\_$top}
and \textit{carbon$\_$bot}, respectively. 

\vspace{0.25cm} \noindent As a consequence of these last four commands, the atoms of the edges will remain
immobile during the simulation (or at least they would if no other command was
applied to them).

\begin{tcolorbox}[colback=mylightblue!5!white,colframe=mylightblue!75!black,title=On imposing a constant velocity to a system]

\vspace{0.25cm} \noindent The \textit{velocity set} commands impose the velocity of a group of atoms at the start of 
a run, but does not enforce the velocity during the entire simulation. 
When \textit{velocity set} is used in combination with \textit{setforce 0 0 0}, 
as is the case here, the atoms
won't feel any force during the entire simulation. According to the Newton equation,
no force means no acceleration, meaning that the initial velocity will persist
during the entire simulation, thus producing a constant velocity motion.
\end{tcolorbox}

\subsection{Data extraction}
Next, in order to measure the strain and stress suffered by the
CNT, let us extract the distance $L$ between
the two edges as well as the force applied on the edges.

\begin{lcverbatim}
variable L equal xcm(carbon_top,z)-xcm(carbon_bot,z)
fix at1 all ave/time 10 10 100 v_L file output_cnt_length.dat
fix at2 all ave/time 10 10 100 f_mysf1[1] f_mysf2[1] &
    file output_edge_force.dat
\end{lcverbatim}

\noindent Let us also add a command to print the atom coordinates in a
\textit{lammpstrj} file every 1000 steps.

\begin{lcverbatim}
dump mydmp all atom 1000 dump.lammpstrj
\end{lcverbatim}

\noindent Finally, let us check the temperature of the non-frozen group over time
by printing it using a \textit{fix ave/time} command:

\begin{lcverbatim}
fix at3 all ave/time 10 10 100 c_Tmid &
    file output_temperature_middle_group.dat
\end{lcverbatim}

\noindent \begin{tcolorbox}[colback=mylightblue!5!white,colframe=mylightblue!75!black,title=About extracting quantity from variable compute or fix]

\vspace{0.25cm} \noindent Notice that the values of the force on each edge are
extracted from the two \textit{fix setforce} \textit{mysf1} and \textit{mysf2}, simply by
calling them using \textit{f$\_$}, the same way variables are called
using \textit{v$\_$} and computes are called using \textit{c$\_$}.
\end{tcolorbox}

\noindent Let us run a small equilibration step to bring the system 
to the required temperature before applying any deformation:

\begin{lcverbatim}
thermo 100
thermo_modify temp Tmid

timestep 1.0
run 5000
\end{lcverbatim}

\noindent With the \textit{thermo$\_$modify} command, we specify to LAMMPS that we
want the temperature $T_\mathrm{mid}$ to be printed in
the terminal, not the temperature of the entire system
(because of the frozen edges, the temperature of
the entire system is not relevant).

\vspace{0.25cm} \noindent Let us impose a constant velocity deformation on the CNT by combining
the \textit{velocity set} command with previously defined \textit{fix setforce}. 
Add the following lines in the \textit{input.lammps} file, 
right after the last \textit{run 5000} command:

\begin{lcverbatim}
# 2*0.0005 A/fs = 0.001 A/fs = 100 m/s
velocity carbon_top set 0 0 0.0005
velocity carbon_bot set 0 0 -0.0005
run 10000
\end{lcverbatim}

\noindent The chosen velocity for the deformation is $100\,\text{m/s}$.
The length $L$ of the CNT increase linearly over
time for $t > 5\,\text{ps}$, as expected from the imposed constant velocity.

\vspace{0.25cm} \noindent [legend-to-add]Figure: Evolution of the length $L$ of the CNT with time.
[legend-to-add]The CNT starts deforming at $t = 5\,\text{ps}$.

\vspace{0.25cm} \noindent The energy, which can be accessed from the log file, shows a non-linear
increase with time once the deformation starts,
which is expected from the typical dependency of bond energy with
bond distance $U_b = k_b \left( r - r_0 \right)^2$.

\vspace{0.25cm} \noindent [legend-to-add]Figure: Evolution of the total energy of the system with time.
[legend-to-add]The CNT starts deforming at $t = 5\,\text{ps}$.

\vspace{0.25cm} \noindent As always, is it important to ensure that the simulation
behaves as expected by opening the \textit{dump.lammpstrj} file with VMD.

\vspace{0.25cm} \noindent [legend-to-add]Figure: CNT before (top) and after (bottom) deformation. See the corresponding \href{https://youtu.be/S05nzreQR18}{video}.

\section{Breakable bonds}
\noindent When using a classical force field, as we just did, the bonds between the atoms 
are non-breakable. Let us perform a similar simulation and deform a small CNT again,
but this time using a reactive force field that allows for the bonds to break
if the applied deformation is large enough.

\subsection{Input file initialization}
\noindent Create a second folder named \textit{breakable-bonds/} next to \textit{unbreakable-bonds/},
and create a new input file in it called \textit{input.lammps}. Type into input.lammps:

\begin{lcverbatim}
# Initialisation
variable T equal 300

units metal
atom_style atomic
boundary p p p
pair_style airebo 2.5 1 1
\end{lcverbatim}

\noindent The first difference with the previous part
is the unit system, here \textit{metal} instead of \textit{real}, a choice
that is imposed by the AIREBO force field. A second difference
is the use of the \textit{atom$\_$style atomic} instead of \textit{molecular},
single no explicit bond information is required with AIREBO.

\begin{tcolorbox}[colback=mylightblue!5!white,colframe=mylightblue!75!black,title=About metal units]

\vspace{0.25cm} \noindent With the \textit{metal} units system of LAMMPS, the time is in pico second, 
distances are in Ångstrom, and the energy is in eV.
\end{tcolorbox}

\subsection{Adapt the topology file}
Since \textit{bond}, \textit{angle}, and \textit{dihedral} do not need to be explicitly
set when using AIREBO, some small changes need to be made to the 
previously generated \textit{.data} file.

\vspace{0.25cm} \noindent Duplicate the previous file \textit{cnt$\_$molecular.data}, name the copy \textit{cnt$\_$atom.data},
place it within \textit{breakable-bonds/}. Then, remove all bond, angle, and dihedral 
information from \textit{cnt$\_$atom.data}. Also, remove the second column of the 
\textit{Atoms} table, so that the \textit{cnt$\_$atom.data} looks like the following: 

\begin{lcverbatim}
700 atoms
1 atom types
-40.000000 40.000000  xlo xhi
-40.000000 40.000000  ylo yhi
-12.130411 67.869589  zlo zhi

Masses

1 12.010700 # CA

Atoms # atomic

1 1 5.162323 0.464617 8.843235 # CA CNT
2 1 4.852682 1.821242 9.111212 # CA CNT
(...)
\end{lcverbatim}

\noindent In addition, remove the \textit{Bonds} table that is placed right after the 
\textit{Atoms} table (near line 743), as well as the \textit{Angles}, \textit{Dihedrals}, 
and \textit{Impropers} tables. The last lines of the file should look like this:

\begin{lcverbatim}
(...)
697 1 4.669892 -2.248901 45.824036 # CA CNT
698 1 5.099893 -0.925494 46.092010 # CA CNT
699 1 5.162323 -0.464617 47.431896 # CA CNT
700 1 5.099893 0.925494 47.699871 # CA CNT
\end{lcverbatim}

\noindent Alternatively, you can also download the file I generate 
by clicking \href{https://lammpstutorials.github.io/lammpstutorials-inputs/level1/breaking-a-carbon-nanotube/breakable-bonds/cnt_atom.data}{here}.

\subsection{Use of AIREBO potential}
\noindent Then, let us import the LAMMPS data file, and set the
pair coefficients by adding the following lines to \textit{input.lammps}

\begin{lcverbatim}
# System definition
read_data cnt_atom.data
pair_coeff * * CH.airebo C
\end{lcverbatim}

\noindent Here, there is one single atom type. We impose this type
to be carbon by using the letter \textit{C}.

\begin{tcolorbox}[colback=mylightblue!5!white,colframe=mylightblue!75!black,title=Setting AIREBO pair coefficients]

\vspace{0.25cm} \noindent In the case of multiple atom types, one has to adapt the \textit{pair$\_$coeff} command. 
If there are 2 atom types, and both are carbon, it would read: \textit{pair$\_$coeff } \textit{ CH.airebo C C}.
If atoms of type 1 are carbon and atoms of type 2 are hydrogen, then \textit{pair$\_$coeff } \textit{ CH.airebo C H}.        
\end{tcolorbox}

\noindent The \textit{CH.airebo} file can be
downloaded by clicking \href{https://lammpstutorials.github.io/lammpstutorials-inputs/level1/breaking-a-carbon-nanotube/breakable-bonds/CH.airebo}{here},
and must be placed within the \textit{breakable-bonds/} folder.
The rest of the \textit{input.lammps} is very similar to the previous one:

\begin{lcverbatim}
change_box all x final -40 40 y final -40 40 z final -60 60

group carbon_atoms type 1
variable carbon_xcm equal -1*xcm(carbon_atoms,x)
variable carbon_ycm equal -1*xcm(carbon_atoms,y)
variable carbon_zcm equal -1*xcm(carbon_atoms,z)
displace_atoms carbon_atoms move ${carbon_xcm} ${carbon_ycm} ${carbon_zcm}

variable zmax equal bound(carbon_atoms,zmax)-0.5
variable zmin equal bound(carbon_atoms,zmin)+0.5
region rtop block INF INF INF INF ${zmax} INF
region rbot block INF INF INF INF INF ${zmin}
region rmid block INF INF INF INF ${zmin} ${zmax}

group carbon_top region rtop
group carbon_bot region rbot
group carbon_mid region rmid

variable zmax_del equal ${zmax}-2
variable zmin_del equal ${zmin}+2
region rdel block INF INF INF INF ${zmin_del} ${zmax_del}
group rdel region rdel
delete_atoms random fraction 0.02 no rdel NULL 482793

reset_atoms id sort yes
velocity carbon_mid create ${T} 48455 mom yes rot yes
fix mynve all nve
compute Tmid carbon_mid temp
fix myber carbon_mid temp/berendsen ${T} ${T} 0.1
fix_modify myber temp Tmid
\end{lcverbatim}

\noindent Note that a large distance of 120 Ångstroms was used for the box size along 
the \textit{z} axis, to allow for larger deformation. In addition, the \textit{change$\_$box} command
was placed before the \textit{displace$\_$atoms} to avoid issue with the 
CNT crossing the edge of the box.

\subsection{Start the simulation}
\noindent Here, let us impose a constant velocity deformation using the atoms
of one edge, while maintaining the other edge fix. Do to so,
one needs to cancel the forces (thus the acceleration) on
the atoms of the edges using the \textit{setforce} command and set
the value of the velocity along the \textit{z} direction.

\vspace{0.25cm} \noindent First, as an equilibration step, let us set the velocity to 0
for the atoms of both edges. Let us fully constrain the edges.
Add the following lines to LAMMPS:

\begin{lcverbatim}
fix mysf1 carbon_bot setforce 0 0 0
fix mysf2 carbon_top setforce 0 0 0
velocity carbon_bot set 0 0 0
velocity carbon_top set 0 0 0

variable L equal xcm(carbon_top,z)-xcm(carbon_bot,z)
fix at1 all ave/time 10 10 100 v_L file output_cnt_length.dat
fix at2 all ave/time 10 10 100 f_mysf1[1] f_mysf2[1] &
    file output_edge_force.dat

dump mydmp all atom 1000 dump.lammpstrj

thermo 100
thermo_modify temp Tmid

timestep 0.0005
run 5000
\end{lcverbatim}

\noindent Note the relatively small timestep of $0.0005\,\text{ps}$
used. A reactive force field usually requires a smaller timestep
than a classical one.
When running \textit{input.lammps} with LAMMPS, you can see that the
temperature deviates from the target temperature of $300\,\text{K}$
at the start of the equilibration, but that
after a few steps, it reaches the target value:

\begin{lcverbatim}
Step  Temp           E_pair         E_mol          TotEng         Press     
0     300           -5084.7276      0             -5058.3973     -1515.7017    
100   237.49462     -5075.4114      0             -5054.5671     -155.05545    
200   238.86589     -5071.9168      0             -5050.9521     -498.15029    
300   220.04074     -5067.1113      0             -5047.7989     -1514.8516    
400   269.23434     -5069.6565      0             -5046.0264     -174.31158    
500   274.92241     -5068.5989      0             -5044.4696     -381.28758    
600   261.91841     -5065.985       0             -5042.9971     -1507.5577    
700   288.47709     -5067.7301      0             -5042.4111     -312.16669    
800   289.85177     -5066.5482      0             -5041.1086     -259.84893    
900   279.34891     -5065.0216      0             -5040.5038     -1390.8508    
1000  312.27343     -5067.6245      0             -5040.217      -465.74352
(...)
\end{lcverbatim}

\subsection{Launch the deformation}
After equilibration, let us set the velocity to 15 m/s and run for
a longer duration than previously. Add the following lines into
\textit{input.lammps}:

\begin{lcverbatim}
# 0.15 A/ps = 15 m/s
velocity carbon_top set 0 0 0.15
run 280000
\end{lcverbatim}

\noindent The CNT should break around step 250000. If not, 
use a longer run. 

\vspace{0.25cm} \noindent When looking at the \textit{lammpstrj} file using VMD, you will see
the bonds breaking. From VMD, use the \textit{DynamicBonds}
representation to properly visualize the bond breaking.

\vspace{0.25cm} \noindent [legend-to-add]Figure: CNT with broken bonds. See the corresponding \href{https://youtu.be/H2_cjoTcVAM}{video}.

\begin{tcolorbox}[colback=mylightblue!5!white,colframe=mylightblue!75!black,title=About bonds in VMD]

\vspace{0.25cm} \noindent Note that VMD guesses bonds based on the distances
between atoms, and not based on the presence of actual
bonds between atoms in the LAMMPS simulation. Therefore the bonds seen
in VMD when using the \textit{DynamicBonds} representation can be misleading.
\end{tcolorbox}

\noindent Looking at the evolution of energy again, one can see that the energy is increasing 
with the deformation, before completely relaxing when the CNT finally breaks.

\vspace{0.25cm} \noindent [legend-to-add]Figure: Evolution of the total energy of the system with time.

\vspace{0.25cm} \noindent You can access the input scripts and data files that
are used in these tutorials from \href{https://github.com/lammpstutorials/lammpstutorials-inputs/}{this Github repository}.
This repository also contains the full solutions to the exercises.

\vspace{0.25cm} \noindent There is a follow-up to this CNT tutorial as \hyperref[mda-label]{MDAnalysis tutorial},
where a post-mortem analysis is performed using Python.

\section{Going further with exercises}
\noindent Each exercise comes with a proposed solution, 
see \hyperref[solutions-label]{Solutions to the exercises}.

\subsection{Plot the strain-stress curves}
\noindent Adapt the current scripts and extract the strain-stress curves for
the two breakable and unbreakable CNTs:

\vspace{0.25cm} \noindent [legend-to-add]Figure: Strain-stain curves for the two CNTs, breakable and unbreakable.

\subsection{Solve the flying ice cube artifact}
\noindent The flying ice cube effect is one of the most famous artifacts of
molecular simulations \cite{wong2016good}.
Download this seemingly simple \href{https://lammpstutorials.github.io/lammpstutorials-inputs/level1/breaking-a-carbon-nanotube/exercises/flying-ice-cube/input.lammps}{input}, which is a simplified
version of the input from the first part of the tutorial.
Run the input with this \href{https://lammpstutorials.github.io/lammpstutorials-inputs/level1/breaking-a-carbon-nanotube/exercises/flying-ice-cube/cnt_molecular.data}{data} file
and this \href{https://lammpstutorials.github.io/lammpstutorials-inputs/level1/breaking-a-carbon-nanotube/exercises/flying-ice-cube/parm.lammps}{parameter} file.

\vspace{0.25cm} \noindent When you run this simulation using LAMMPS, you should see that the temperature is
very close to $300\,\text{K}$, as expected.

\begin{lcverbatim}
Step   Temp        E_pair      E_mol       TotEng      Press     
0      327.4142    589.20707   1980.6012   3242.2444   60.344754    
1000   300.00184   588.90015   1980.9013   3185.9386   51.695282
(...)
\end{lcverbatim}

\noindent However, if you look at the system using VMD, the atoms are not moving.

\vspace{0.25cm} \noindent Can you identify the origin of the issue, and fix the input?

\subsection{Insert gas in the carbon nanotube}
\noindent Modify the input from the unbreakable CNT, and add atoms of argon
within the CNT. 

\vspace{0.25cm} \noindent Use the following \textit{pair$\_$coeff} for the argon,
and a mass of \textit{39.948}:

\begin{lcverbatim}
pair_coeff 2 2 0.232 3.3952 
\end{lcverbatim}

\noindent [legend-to-add]Figure: Argon atoms in a CNT.  See the corresponding \href{https://www.youtube.com/watch?v=J4z_fZK7ekA}{video}.

\subsection{Make a membrane of CNTs}
\noindent Replicate the CNT along the \textit{x}
and \textit{y} direction, and equilibrate the system to 
create an infinite membrane made of multiple CNTs. 

\vspace{0.25cm} \noindent Apply a shear deformation along \textit{xy}.

\vspace{0.25cm} \noindent [legend-to-add]Figure: Multiple carbon nanotubes forming a membrane.  

\begin{tcolorbox}[colback=mylightblue!5!white,colframe=mylightblue!75!black,title=Hint]

\vspace{0.25cm} \noindent The box must be converted to triclinic to support deformation
along \textit{xy}.
\end{tcolorbox}


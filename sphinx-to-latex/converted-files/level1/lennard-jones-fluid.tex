\chapter{Lennard-Jones fluid}
\label{lennard-jones-label}

\noindent \vspace{-1cm} \noindent \textcolor{graytitle}{\textit{{\Large The very basics of LAMMPS through a simple example}}\vspace{0.5cm} }

\vspace{0.25cm} \noindent The objective of this tutorial is to perform
the simulation of a binary fluid using LAMMPS.

\vspace{0.25cm} \noindent The system is a Lennard-Jones fluid made of neutral
particles with two different diameters in a cubic box with periodic
boundary conditions. The temperature of the system is maintained
using a Langevin thermostat \cite{schneider1978molecular}, and
basic quantities are extracted
from the system, including the potential and kinetic energies. 

\vspace{0.25cm} \noindent This tutorial illustrates several key ingredients of
molecular dynamics simulations, such as system initialization,
energy minimization, integration of the equations of motion,
and trajectory visualization.

\section{My first input}
\noindent To run a simulation using LAMMPS, one needs to
write a series of commands in an input script. For clarity,
this script will be divided into five categories which we are going to
fill up one by one. 

\vspace{0.25cm} \noindent Create a folder, call it \textit{my-first-input/}, and then create a blank
text file in it called \textit{input.lammps}. Copy the following lines
in \textit{input.lammps}, where a line starting with a brace ($\#$)
is a comment that is ignored by LAMMPS:

\begin{lcverbatim}
# PART A - ENERGY MINIMIZATION
# 1) Initialization
# 2) System definition
# 3) Simulation settings
# 4) Visualization
# 5) Run
\end{lcverbatim}

\noindent These five categories are not required in every
input script, and should not necessarily be in that
exact order. For instance, parts 3 and 4 could be inverted, or
part 4 could be omitted. Note however that LAMMPS reads input
files from top to bottom, therefore the \textit{Initialization} and 
\textit{System definition} categories must appear at the top of the
input, and the \textit{Run} category at the bottom.

\subsection{System initialization}
\noindent In the first section of the script, called \textit{Initialization},
let us indicate to LAMMPS the most basic information
about the simulation, such as:

\begin{itemize}
\item the conditions at the boundaries of the box (e.g. periodic or non-periodic),
\item the type of atoms (e.g. uncharged single dots or spheres with angular velocities).
\end{itemize}

\vspace{0.25cm} \noindent Enter the following lines in \textit{input.lammps}:

\begin{lcverbatim}
# 1) Initialization
units lj
dimension 3
atom_style atomic
pair_style lj/cut 2.5
boundary p p p
\end{lcverbatim}

\noindent The first line, \textit{units lj}, indicates that we want to
use the system of unit called \textit{LJ}, for Lennard-Jones, for
which all quantities are unitless. 

\begin{tcolorbox}[colback=mylightblue!5!white,colframe=mylightblue!75!black,title=About Lennard-Jones (LJ) units]

\vspace{0.25cm} \noindent Lennard-Jones (LJ) units are a dimensionless system of units.
LJ units are often used in molecular simulations
and theoretical calculations. When using LJ units:

\begin{itemize}
\item energies are expressed in units of $\epsilon$, where $\epsilon$ is the
  depth of the potential of the LJ interaction,
\item distances are expressed in units of $\sigma$, where $\sigma$ is the distance
  at which the particle-particle potential energy is zero,
\item masses are expressed in units of the atomic mass $m$.
\end{itemize}

\vspace{0.25cm} \noindent All the other quantities are normalized by a combination of $\epsilon$, $\sigma$,
and $m$. For instance, time is expressed in units of $\sqrt{ \epsilon / m \sigma^2}$.
Find details on the \href{https://docs.lammps.org/units.html}{LAMMPS website}.
\end{tcolorbox}

\noindent The second line, \textit{dimension 3}, indicates that the simulation
is 3D. The third line, \textit{atom$\_$style atomic}, that the \textit{atomic} style
will be used, therefore each atom is just a dot with a mass.

\begin{tcolorbox}[colback=mylightblue!5!white,colframe=mylightblue!75!black,title=About the atom style]

\vspace{0.25cm} \noindent While we are keeping things as simple as possible in this tutorial,
different \textit{atom$\_$style} will be used in the following tutorials.
Notably, these other atom styles will allow us to create molecules,
i.e. atoms with partial charges and chemical bonds. You can find the complete list
of implemented atom styles from the \href{https://docs.lammps.org/atom_style.html}{atom style page}.
\end{tcolorbox}

\noindent The fourth line, \textit{pair$\_$style lj/cut 2.5}, indicates that atoms
will be interacting through a Lennard-Jones potential with
a cut-off equal to $r_c = 2.5$ (unitless)
\cite{wang2020lennard,fischer2023history}:

$$E_{ij} (r) = 4 \epsilon_{ij} \left[ \left( \dfrac{\sigma_{ij}}{r} \right)^{12} - \left( \dfrac{\sigma_{ij}}{r} \right)^{6} \right], ~ \text{for} ~ r < r_c,$$
where $r$ is the inter-particles distance,
$\epsilon_{ij}$ the depth of potential well that sets the interaction strength, and
$\sigma_{ij}$ the distance parameter, or particle effective size.
Here, the indexes \textit{ij} refer to the particle types \textit{i} and \textit{j}.

\begin{tcolorbox}[colback=mylightblue!5!white,colframe=mylightblue!75!black,title=About Lennard-Jones potential]

\vspace{0.25cm} \noindent The Lennard-Jones potential offers a simplified representation
that captures the fundamental
aspects of interactions among atoms. It depicts a scenario where two
particles exhibit repulsion at extremely close distances, attraction at moderate
distances, and no interaction at infinite separation. The repulsive part of the 
Lennard-Jones potential (i.e. the term $\propto r^{-12}$) is associated
with the Pauli exclusion principle. The attractive part (i.e. the term
in $\propto - r^{-6}$)
is linked with the London dispersion forces.
\end{tcolorbox}

\noindent The last line, \textit{boundary p p p}, indicates that the
periodic boundary conditions will be used along all three
directions of space (the 3 \textit{p} stand for \textit{x}, \textit{y}, and \textit{z},
respectively).

\vspace{0.25cm} \noindent At this point, the \textit{input.lammps} is a 
LAMMPS input script that does nothing.
You can run it using LAMMPS to verify that the \textit{input} contains
no mistake by running the following command in the terminal
from the \textit{my-first-input/}  folder:

\begin{lcverbatim}
lmp -in input.lammps
\end{lcverbatim}

\noindent Here \textit{lmp} is linked to my compiled LAMMPS version.
Running the previous command should return:

\begin{lcverbatim}
LAMMPS (2 Aug 2023 - Update 1)
Total wall time: 0:00:00
\end{lcverbatim}

\noindent In case there is a mistake in the input script, for example, if
\textit{atom$\_$stile} is written instead of \textit{atom$\_$style}, LAMMPS
gives you an explicit warning:

\begin{lcverbatim}
LAMMPS (2 Aug 2023 - Update 1)
ERROR: Unknown command: atom_stile  atomic (src/input.cpp:232)
Last command: atom_stile atomic
\end{lcverbatim}

\subsection{System definition}
Let us fill the \textit{System definition} category of the input script:

\begin{lcverbatim}
# 2) System definition
region simulation_box block -20 20 -20 20 -20 20
create_box 2 simulation_box
create_atoms 1 random 1500 341341 simulation_box
create_atoms 2 random 100 127569 simulation_box
\end{lcverbatim}

\noindent The first line, \textit{region simulation$\_$box (...)}, creates a region
named \textit{simulation$\_$box} that is a block (i.e. a rectangular cuboid) that
extends from -20 to 20 (no unit) along all 3 directions of space.

\vspace{0.25cm} \noindent The second line, \textit{create$\_$box 2 simulation$\_$box}, creates a simulation box based on
the region \textit{simulation$\_$box} with \textit{2} types of atoms.

\vspace{0.25cm} \noindent The third line, \textit{create$\_$atoms (...)} creates 1500 atoms of type 1
randomly within the region \textit{simulation$\_$box}. The integer \textit{341341} is a
seed that can be changed in order to create different
initial conditions for the simulation. The fourth line
creates 100 atoms of type 2.

\vspace{0.25cm} \noindent If you run LAMMPS, you should see the following information in the
terminal:

\begin{lcverbatim}
(...)
Created orthogonal box = (-20 -20 -20) to (20 20 20)
(...)
Created 1500 atoms
(...)
Created 100 atoms
(...)
\end{lcverbatim}

\noindent From what is printed in the terminal, it is clear that
LAMMPS correctly interpreted the commands, and first created
the box with desired dimensions, then 1500 atoms, and then 100
atoms.

\subsection{Simulation Settings}
\noindent Let us fill the \textit{Simulation Settings} category section of
the \textit{input} script:

\begin{lcverbatim}
# 3) Simulation settings
mass 1 1
mass 2 1
pair_coeff 1 1 1.0 1.0
pair_coeff 2 2 0.5 3.0
\end{lcverbatim}

\noindent The two first commands, \textit{mass (...)}, attribute a mass
equal to 1 (unitless) to both atoms of type 1 and 2.
Alternatively, one could have written
these two commands into one single line: \textit{mass $\star 1$},
where the star symbol means \textit{all} the atom types of the simulation. 

\vspace{0.25cm} \noindent The third line, \textit{pair$\_$coeff 1 1 1.0 1.0}, sets the Lennard-Jones
coefficients for the interactions between atoms of type 1,
respectively the energy parameter
$\epsilon_{11} = 1.0$ and the distance parameter $\sigma_{11} = 1.0$. 

\vspace{0.25cm} \noindent Similarly, the last line sets the Lennard-Jones coefficients for
the interactions between atoms of type 2, $\epsilon_{22} = 0.5$,
and $\sigma_{22} = 3.0$.

\begin{tcolorbox}[colback=mylightblue!5!white,colframe=mylightblue!75!black,title=About cross parameters]

\vspace{0.25cm} \noindent By default, LAMMPS calculates the cross coefficients between the different atom types
using geometric average: 
$\epsilon_{ij} = \sqrt{\epsilon_{ii} \epsilon_{jj}}$,
$\sigma_{ij} = \sqrt{\sigma_{ii} \sigma_{jj}}$. 
In the present case, and even without specifying it explicitly, we thus have:

\begin{itemize}
\item $\epsilon_{12} = \sqrt{1.0 \times 0.5} = 0.707$, and 
\item $\sigma_{12} = \sqrt{1.0 \times 3.0} = 1.732$.
\end{itemize}

\vspace{0.25cm} \noindent When necessary, cross-parameters can be explicitly specified
by adding the following line to the input file: \textit{pair$\_$coeff 1 2 0.707 1.732}. 
This can be used for instance to increase the attraction between particles
of type 1 and 2, without affecting the interactions between particles of the same type.

\vspace{0.25cm} \noindent Note that the arithmetic rule, also known as 
Lorentz-Berthelot rule :cite:`lorentz1881ueber,berthelot1898melange`, where 
$\epsilon_{ij} = \sqrt{\epsilon_{ii} \epsilon_{jj}}$,
$\sigma_{ij} = (\sigma_{ii}+\sigma_{jj})/2$, 
is more common than the geometric rule. However, neither the geometric nor the
arithmetic rules are based on rigorous arguments, so here
the geometric rule will do just fine. 
\end{tcolorbox}

\noindent Due to the chosen Lennard-Jones parameters, the two types of particles
are given different effective diameters, as can be seen by plotting
$E_{11} (r)$, 
$E_{12} (r)$,
and $E_{22} (r)$.

\vspace{0.25cm} \noindent [legend-to-add]Figure: The Lennard-Jones potential $E_{ij} (r)$
[legend-to-add]as a function of the inter-particle distance, where
[legend-to-add]$i, j = 1 ~ \text{or} ~ 2$. This figure was generated using Python
[legend-to-add]with Matplotlib Pyplot, and the notebook can be accessed \href{https://github.com/lammpstutorials/lammpstutorials.github.io/blob/version2.0/docs/sphinx/source/tutorials/figures/level1/lennard-jones-fluid/lennard-jones-pyplot.ipynb}{from Github}.
[legend-to-add]The Pyplot parameters used for all figures can be accessed in a \href{https://github.com/simongravelle/pyplot-perso}{dedicated repository}.

\subsection{Energy minimization}
\noindent The system is now fully parametrized. Let us fill the two last remaining sections
by adding the following lines to \textit{input.lammps}:

\begin{lcverbatim}
# 4) Visualization
thermo 10
thermo_style custom step temp pe ke etotal press

# 5) Run
minimize 1.0e-4 1.0e-6 1000 10000
\end{lcverbatim}

\noindent The \textit{thermo} command asks LAMMPS to print
thermodynamic information (e.g. temperature, energy) in the
terminal every given number of steps, here 10 steps. 
The \textit{thermo$\_$style custom} requires LAMMPS to print 
the system temperature (\textit{temp}), potential energy (\textit{pe}),
kinetic energy (\textit{ke}), total energy (\textit{etotal}),
and pressure (\textit{press}). Finally, the \textit{minimize} line
asks LAMMPS to perform an energy minimization of the system.

\begin{tcolorbox}[colback=mylightblue!5!white,colframe=mylightblue!75!black,title=About energy minimization]

\vspace{0.25cm} \noindent An energy minimization procedure consists of adjusting
the coordinates of the atoms that are too close to each other until one of the stopping
criteria is reached. By default, LAMMPS uses the conjugate
gradient (CG) algorithm :cite:`hestenes1952methods` (see all the other
implemented methods on the \href{https://docs.lammps.org/min_style.html}{min style} page), which runs 
until one of the following criteria is reached:

\begin{itemize}
\item The change in energy between two iterations is less than 1.0e-4.
\item The maximum force between two atoms in the system is lower than 1.0e-6.
\item The maximum number of iterations is 1000.
\item The maximum number of times the force and the energy have been evaluated is 10000.
\end{itemize}
\end{tcolorbox}

\noindent Now running the simulation, we can see how the thermodynamic
variables evolve as the simulation progresses:

\begin{lcverbatim}
Step  Temp  PotEng         KinEng    TotEng         Press     
0     0     78840982       0         78840982       7884122      
10    0     169.90532      0         169.90532      17.187291    
20    0    -0.22335386     0        -0.22335386    -0.0034892297 
30    0    -0.31178296     0        -0.31178296    -0.0027290466 
40    0    -0.38135002     0        -0.38135002    -0.0016419218 
50    0    -0.42686621     0        -0.42686621    -0.0015219081 
60    0    -0.46153953     0        -0.46153953    -0.0010659992 
70    0    -0.48581568     0        -0.48581568    -0.0014849169 
80    0    -0.51799572     0        -0.51799572    -0.0012995545 
(...)
\end{lcverbatim}

\noindent These lines give us information about
the progress of the energy minimization. First, at the start
of the simulation (Step 0), the energy in the system is
huge: 78840982 (unitless). This was expected because
the atoms have been created at random positions within the
simulation box and some of them are probably overlapping,
resulting in a large initial energy which is the consequence
of the repulsive part of the Lennard-Jones interaction
potential. As the energy minimization progresses, the energy
rapidly decreases and reaches a negative value, indicating that the atoms have been
displaced at reasonable distances from each other.

\begin{tcolorbox}[colback=mylightblue!5!white,colframe=mylightblue!75!black,title=On the temperature during energy minimization]

\vspace{0.25cm} \noindent As a side note, during energy minimization both temperature and kinetic energy remain equal to
their initial values of 0. This is expected as the conjugate gradient
algorithm only affects the positions of the particles based on the
forces between them, without affecting their velocities.
\end{tcolorbox}

\noindent Other useful information has been printed in the terminal, for
example, LAMMPS tells us that the first of the four criteria
to be satisfied was the energy:

\begin{lcverbatim}
Minimization stats:
Stopping criterion = energy tolerance
\end{lcverbatim}

\subsection{Molecular dynamics}
The system is now ready. Let us continue filling up the
input script and adding commands to perform a molecular dynamics
simulation that will start from the final state of the previous energy
minimization step.

\begin{tcolorbox}[colback=mylightblue!5!white,colframe=mylightblue!75!black,title=Background Information -- What is molecular dynamics?]

\vspace{0.25cm} \noindent Molecular dynamics (MD) is based on the numerical solution of the Newtonian
equations of motion for every atom $i$,

$$\sum_{j \ne i} \boldsymbol{F}_{ji} = m_i \times \boldsymbol{a}_i,$$
\end{tcolorbox}

\noindent In the same input script, after the \textit{minimization} command, add the following
lines:

\begin{lcverbatim}
# PART B - MOLECULAR DYNAMICS
# 4) Visualization
thermo 50
\end{lcverbatim}

\noindent Since LAMMPS reads the input from top to bottom, these lines will be
executed after the energy minimization. There is no need to re-initialize
or re-define the system. The \textit{thermo} command is called a second time within
the same input, so the previously entered value of 10 will be replaced by
the value of 50 as soon as \textit{PART B} starts.

\vspace{0.25cm} \noindent Then, let us add a second \textit{Run} section:

\begin{lcverbatim}
# 5) Run
fix mynve all nve
fix mylgv all langevin 1.0 1.0 0.1 1530917
timestep 0.005
run 10000
\end{lcverbatim}

\noindent The \textit{fix nve} is used to update the positions and the velocities of the
atoms in the group \textit{all} at every step. The group \textit{all} is a default group
that contains every atom.

\vspace{0.25cm} \noindent The second fix applies a Langevin thermostat to the atoms of the group
\textit{all}, with a desired initial temperature of 1.0 (unitless), and a final
temperature of 1.0 as well \cite{schneider1978molecular}.
A \textit{damping} parameter of 0.1 is used. The \textit{damping}
parameter determines how rapidly the temperature is relaxed to its desired value.
The number \textit{1530917} is a seed, you can
change it to perform statistically independent simulations.
Finally, we choose the value of the \textit{timestep} and we ask LAMMPS to
run for 10000 steps, corresponding to a total duration of 50 (unitless).

\begin{tcolorbox}[colback=mylightblue!5!white,colframe=mylightblue!75!black,title=What is a fix?]

\vspace{0.25cm} \noindent In LAMMPS, a \textit{fix} is a command that performs specific tasks during a simulation,
such as imposing constraints, applying forces, or modifying particle properties.
Other LAMMPS-specific terms are defined in the Tutorial\,\ref{glossary-label}.
\end{tcolorbox}

\noindent After running the simulation, similar lines should appear in the terminal:

\begin{lcverbatim}
Step   Temp          PotEng         KinEng       TotEng        Press     
388    0             -0.95476642    0           -0.95476642   -0.000304834
400    0.68476875    -0.90831467    1.0265112    0.11819648    0.023794293  
500    0.97168188    -0.56803405    1.4566119    0.88857783    0.02383215   
600    1.0364167     -0.44295618    1.5536534    1.1106972     0.027985679  
700    1.010934      -0.39601767    1.5154533    1.1194356     0.023064983  
800    0.98641731    -0.37866057    1.4787012    1.1000406     0.023131153  
900    1.0074571     -0.34951264    1.5102412    1.1607285     0.023520785 
(...)
\end{lcverbatim}

\noindent The second column shows that the temperature \textit{Temp}
starts from 0, but rapidly reaches the
requested value and stabilize itself near $T=1$. 

\vspace{0.25cm} \noindent From what has been printed in the \textit{log} file, one can
plot the potential energy ($p_\text{e}$)
and the kinetic energy ($k_\text{e}$) of
the system over time (see the figure below).

\vspace{0.25cm} \noindent [legend-to-add]Figure: a) The potential energy ($p_\text{e}$) rapidly decreases during
[legend-to-add]energy minimization (orange). Then, after the molecular dynamics simulation starts,
[legend-to-add]$p_\text{e}$ increases until it reaches a plateau value of about -0.25 (blue). 
[legend-to-add]b) The kinetic energy ($k_\text{e}$) is equal to zero during energy minimization,
[legend-to-add]then increases during molecular dynamics until it reaches a plateau value of about 1.5.

\subsection{Trajectory visualization}
\noindent The simulation is running well, but we would like to
visualize the trajectories of the atoms. To do so, we first need
to print the positions of the atoms in a file at a regular interval.

\vspace{0.25cm} \noindent Add the following command to the \textit{input.lammps} file,
in the \textit{Visualization} section of \textit{PART B}:

\begin{lcverbatim}
dump mydmp all atom 100 dump.lammpstrj
\end{lcverbatim}

\noindent Run the \textit{input.lammps} using LAMMPS again. A file named \textit{dump.lammpstrj}
must appear within \textit{my-first-input/}. A \textit{.lammpstrj} file can
be opened using VMD. With Ubuntu/Linux, you can simply execute in the terminal:

\begin{lcverbatim}
vmd dump.lammpstrj
\end{lcverbatim}

\noindent Otherwise, you can open VMD and import the \textit{dump.lammpstrj}
file manually using \textit{File -> New molecule}.

\vspace{0.25cm} \noindent By default, you should see a cloud
of lines, but you can improve the representation (see this \hyperref[vmd-label]{VMD tutorial}
for basic instructions).

\vspace{0.25cm} \noindent [legend-to-add]Figure: View of a slice of the system using VMD, with both
[legend-to-add]types of atoms represented as spheres. See the corresponding \href{https://youtu.be/vdSIJM5fVJE}{video}.

\section{Improving the script}
\noindent Let us improve the input script and perform slightly more
advanced operations, such as imposing a specific initial
positions to the atoms, and restarting the simulation
from a previously saved configuration. 

\subsection{Control the initial atom positions}
\noindent Create a new folder next to \textit{my-first-input/}, and call
it \textit{improved-input/}. Then, create a new input file within \textit{improved-input/}
and call it \textit{input.min.lammps}.

\vspace{0.25cm} \noindent Similarly to what has been done previously, copy the following lines
into \textit{input.min.lammps}:

\begin{lcverbatim}
# 1) Initialization
units lj
dimension 3
atom_style atomic
pair_style lj/cut 2.5
boundary p p p
\end{lcverbatim}

\noindent To create the atoms of types 1 and 2 in two separate
regions, let us create three separate regions: A cubic region
for the simulation box and two additional regions for placing the atoms:

\begin{lcverbatim}
# 2) System definition
region simulation_box block -20 20 -20 20 -20 20
create_box 2 simulation_box
region region_cylinder_in cylinder z 0 0 10 INF INF side in
region region_cylinder_out cylinder z 0 0 10 INF INF side out
create_atoms 1 random 1000 341341 region_cylinder_out
create_atoms 2 random 150 127569 region_cylinder_in
\end{lcverbatim}

\noindent The \textit{side in} and \textit{side out} keywords
are used to define regions that are respectively inside
and outside of the cylinder of radius 10. Then, copy similar lines
as previously into \textit{input.min.lammps}:

\begin{lcverbatim}
# 3) Simulation settings
mass 1 1
mass 2 1
pair_coeff 1 1 1.0 1.0
pair_coeff 2 2 0.5 3.0

# 4) Visualization
thermo 10
thermo_style custom step temp pe ke etotal press
dump mydmp all atom 10 dump.min.lammpstrj

# 5) Run
minimize 1.0e-4 1.0e-6 1000 10000
write_data minimized_coordinate.data
\end{lcverbatim}

\noindent The main novelty, compared to the previous input script, is the \textit{write$\_$data}
command. This command is used to print the final state of the simulation in
a file named \textit{minimized$\_$coordinate.data}. Note that the \textit{write$\_$data} command
is placed after the \textit{minimize} command. This \textit{.data} file will be used later
to restart the simulation from the final state of the energy minimization step.

\vspace{0.25cm} \noindent Run the \textit{input.min.lammps} script using LAMMPS. A new dump file named
\textit{dump.min.lammpstrj} will appear in the folder, allowing you to visualize
the atom's trajectories during minimization. In
addition, a file named \textit{minimized$\_$coordinate.data} will be created. 

\vspace{0.25cm} \noindent If you open \textit{minimized$\_$coordinate.data} with a text editor, you can see
that it contains all the information necessary to restart the
simulation, such as the number of atoms and the box size, the
\textit{masses}, the \textit{pair$\_$coeffs}:

\begin{lcverbatim}
1150 atoms
2 atom types

-20 20 xlo xhi
-20 20 ylo yhi
-20 20 zlo zhi

Masses

1 1
2 1

Pair Coeffs # lj/cut

1 1 1
2 0.5 3
(...)
\end{lcverbatim}

\noindent The \textit{minimized$\_$coordinate.data} file also contains the final
positions of the atoms:

\begin{lcverbatim}
(...)
Atoms # atomic

970 1 4.4615279184230525 -19.88248310680258 -19.497251754277872 0 0 0
798 1 1.0773937287460968 -17.57843015813612 -19.353475858951473 0 0 0
21 1 -17.542385434367777 -16.647460269156497 -18.93914807895693 0 0 0
108 1 -15.96241088290946 -15.956274144833264 -19.016419910024062 0 0 0
351 1 0.08197850837343444 -16.852380573900156 -19.28249747472579 0 0 0
402 1 -5.270160783673711 -15.592291204068946 -19.6382667867645 0 0 0
(...)
\end{lcverbatim}

\noindent The first five columns of the \textit{Atoms} section
correspond (from left to right) to the atom indexes (from 1
to the total number of atoms, 1150), the atom types (1 or 2
here), and the atoms positions $x$, $y$, $z$.
The last three columns are image flags that keep track of which
atoms crossed the periodic boundary.

\subsection{Restarting from a saved configuration}
\noindent Let us create a new input file and start a
molecular dynamics simulation directly from the previously
saved configuration. Within \textit{improved-input/}, create a new file
named \textit{input.md.lammps} and copy the same lines as previously:

\begin{lcverbatim}
# 1) Initialization
units lj
dimension 3
atom_style atomic
pair_style lj/cut 2.5
boundary p p p
\end{lcverbatim}

\noindent Now, instead of creating a new region and adding atoms to it, we
can simply add the following command:

\begin{lcverbatim}
# 2) System definition
read_data minimized_coordinate.data
\end{lcverbatim}

\noindent By visualizing the previously generated \textit{dump.min.lammpstrj}
file, you may have noticed that some atoms have moved from
one region to the other during minimization.
To start the simulation from a clean slate, with
only atoms of type 2 within the cylinder and atoms of type
1 outside the cylinder, let us delete the misplaced atoms
by adding the following commands to \textit{input.md.lammps}:

\begin{lcverbatim}
read_data minimized_coordinate.data
region region_cylinder_in cylinder z 0 0 10 INF INF side in
region region_cylinder_out cylinder z 0 0 10 INF INF side out
group group_type_1 type 1
group group_type_2 type 2
group group_region_in region region_cylinder_in
group group_region_out region region_cylinder_out
group group_type_1_in intersect group_type_1 group_region_in
group group_type_2_out intersect group_type_2 group_region_out
delete_atoms group group_type_1_in
delete_atoms group group_type_2_out
\end{lcverbatim}

\noindent The two first \textit{region} commands recreate
the previously defined regions, which is necessary since
regions are not saved by the \textit{write$\_$data} command.

\vspace{0.25cm} \noindent The first two \textit{group} commands create atom groups based on their types.
The next two \textit{group} commands create atom groups based on their
positions at the beginning of the simulation, i.e. when the commands
are being read by LAMMPS.
The last two \textit{group} commands create atom groups based on the intersection
between the previously defined groups.

\vspace{0.25cm} \noindent Finally, the two \textit{delete$\_$atoms} commands delete the
atoms of type 1 that are located within the cylinder, as
well as the atoms of type 2 that are located outside the
cylinder, respectively. 

\vspace{0.25cm} \noindent When you run the \textit{input.md.lammps} input using LAMMPS, you
can see in the \textit{log} file how many atoms are in each group,
and how many atoms have been deleted:

\begin{lcverbatim}
1000 atoms in group group_type_1
150 atoms in group group_type_2
149 atoms in group group_region_in
1001 atoms in group group_region_out
0 atoms in group group_type_1_in
1 atoms in group group_type_2_out
Deleted 0 atoms, new total = 1150
Deleted 1 atoms, new total = 1149
\end{lcverbatim}

\noindent Add the following lines to \textit{input.md.lammps}.
Note the absence of \textit{Simulation settings} section,
because the settings are taken from the \textit{.data} file.

\begin{lcverbatim}
# 4) Visualization
thermo 1000
dump mydmp all atom 1000 dump.md.lammpstrj
\end{lcverbatim}

\noindent Let us extract the number of atoms of each type
inside the cylinder as a function of time, by
adding the following commands to \textit{input.md.lammps}:

\begin{lcverbatim}
variable number_type1_in equal count(group_type_1,region_cylinder_in)
variable number_type2_in equal count(group_type_2,region_cylinder_in)
fix myat1 all ave/time 10 200 2000 v_number_type1_in &
    file output-population1vstime.dat
fix myat2 all ave/time 10 200 2000 v_number_type2_in &
    file output-population2vstime.dat
\end{lcverbatim}

\noindent The 2 \textit{variables} are used to count
the number of atoms of a specific group in the \textit{region$\_$cylinder$\_$in} region. 

\vspace{0.25cm} \noindent The two \textit{fix ave/time}
are calling the previously defined variables and are printing
their values into text files.
By using \textit{10 200 2000}, variables are evaluated every 10 steps, 
averaged 200 times, and printed in the \textit{.dat} files every 2000 steps.

\vspace{0.25cm} \noindent Let us also extract the coordination number per atom between atoms 
of type 1 and 2, i.e. the average number of atoms of type 2 in the vicinity 
of the atoms of type 1. This coordination number will be used as
an indicator of the degree of mixing of our binary mixture. 
Add the following lines into \textit{input.md.lammps}:

\begin{lcverbatim}
compute coor12 group_type_1 coord/atom cutoff 2.0 group group_type_2
compute sumcoor12 all reduce ave c_coor12
fix myat3 all ave/time 10 200 2000 &
    c_sumcoor12 file coordinationnumber12.dat
\end{lcverbatim}

\noindent The \textit{compute ave} is used to average the per atom
coordination number that is calculated by the \textit{coord/atom} compute.
This averaging is necessary as \textit{coord/atom} returns an array where each value corresponds 
to a certain couple of atoms i-j. Such an array can't be printed by \textit{fix ave/time}. 
Finally, let us complete the script by adding the following lines 
to \textit{input.md.lammps}:

\begin{lcverbatim}
# 5) Run
velocity all create 1.0 4928459 mom yes rot yes dist gaussian
fix mynve all nve
fix mylgv all langevin 1.0 1.0 0.1 1530917 zero yes
timestep 0.005
run 300000
write_data mixed.data
\end{lcverbatim}

\noindent There are a few differences from the previous simulation.
First, the \textit{velocity create}
command attributes an initial velocity to every atom.
The initial velocity is chosen so that the average initial
temperature is equal to 1 (unitless). The additional
keywords ensure that no linear momentum (\textit{mom yes}) and no angular
momentum (\textit{rot yes}) are given to the system and that the generated
velocities are distributed as a Gaussian. Another improvement
is the \textit{zero yes} keyword in the Langevin thermostat, which
ensures that the total random force is equal to zero.

\vspace{0.25cm} \noindent Run \textit{input.md.lammps} using LAMMPS and visualize the trajectory
using VMD:

\vspace{0.25cm} \noindent [legend-to-add]Figure: Evolution of the atom populations during mixing.

\vspace{0.25cm} \noindent After running \textit{input.md.lammps} using LAMMPS, you can observe the number
of atoms in each region from the generated data files, as
well as the evolution of the coordination number due to mixing:

\vspace{0.25cm} \noindent [legend-to-add]Figure: Evolution of the number of atoms within the \textit{region$\_$cylinder$\_$in} region
[legend-to-add]as a function of time (a), and evolution of the coordination number (b). 

\vspace{0.25cm} \noindent You can access the input scripts and data files that
are used in these tutorials from \href{https://github.com/lammpstutorials/lammpstutorials-inputs/}{this Github repository}.
This repository also contains the full solutions to the exercises.

\section{Going further with exercises}
\noindent Each exercise comes with a proposed solution, 
see \hyperref[solutions-label]{Solutions to the exercises}.

\subsection{Solve Lost atoms error}
\noindent For this exercise, the following input script is provided:

\begin{lcverbatim}
units lj
dimension 3
atom_style atomic
pair_style lj/cut 2.5
boundary p p p

region simulation_box block -20 20 -20 20 -20 20
create_box 1 simulation_box
create_atoms 1 random 1000 341841 simulation_box

mass 1 1
pair_coeff 1 1 1.0 1.0

dump mydmp all atom 100 dump.lammpstrj
thermo 100
thermo_style custom step temp pe ke etotal press

fix mynve all nve
fix mylgv all langevin 1.0 1.0 0.1 1530917
timestep 0.005

run 10000
\end{lcverbatim}

\noindent As it is, this input returns one of the most common
error that you will encounter using LAMMPS:

\begin{lcverbatim}
ERROR: Lost atoms: original 1000 current 984
\end{lcverbatim}

\noindent The goal of this exercise is to fix the \textit{Lost atoms} error without 
using any other command than the ones already present. You can 
only play with the values of the parameters and/or replicate every
command as many times as needed.

\begin{tcolorbox}[colback=mylightblue!5!white,colframe=mylightblue!75!black,title=Note]

\vspace{0.25cm} \noindent This script is failing because particles are created
randomly in space, some of them are likely overlapping,
and no energy minimization is performed prior
to start the molecular dynamics simulation.
\end{tcolorbox}

\subsection{Create a demixed dense phase}
Starting from one of the \textit{input} created in this tutorial,
fine-tune the parameters such as particle numbers and interaction
to create a simulation with the following properties:

\begin{itemize}
\item the density in particles must be high,
\item both particles of type 1 and 2 must have the same size,
\item particles of type 1 and 2 must demix. 
\end{itemize}

\vspace{0.25cm} \noindent [legend-to-add]Figure: Snapshots taken at different times showing the particles of type 1 
[legend-to-add]and type 2 progressively demixing and forming large demixed areas.  

\begin{tcolorbox}[colback=mylightblue!5!white,colframe=mylightblue!75!black,title=Hint]

\vspace{0.25cm} \noindent An easy way to create a dense phase is to allow the box dimensions 
to relax until the vacuum disappears. You can do that 
by replacing the \textit{fix nve} with \textit{fix nph}.
\end{tcolorbox}

\subsection{From atoms to molecules}
Add a bond between particles of \textit{type 2} to create
dumbbell molecules instead of single particles.

\vspace{0.25cm} \noindent [legend-to-add]Figure: Dumbbell molecules made of 2 large spheres
[legend-to-add]mixed with smaller particles (small spheres). 
[legend-to-add]See the corresponding \href{https://youtu.be/R_oHonOQi68}{video}.

\vspace{0.25cm} \noindent Similarly to the dumbbell molecules, create a small polymer,
i.e. a long chain of particles linked by bonds and angles.

\vspace{0.25cm} \noindent [legend-to-add]Figure: A single small polymer molecule made of
[legend-to-add]9 large spheres mixed with smaller particles. 
[legend-to-add]See the corresponding \href{https://youtu.be/LfqcfP3ZQcY}{video}.

\begin{tcolorbox}[colback=mylightblue!5!white,colframe=mylightblue!75!black,title=Hints]

\vspace{0.25cm} \noindent Use a \textit{molecule template} to easily insert as many atoms connected
by bonds (i.e. molecules) as you want. A molecule 
template typically begins as follows:

\begin{lcverbatim}
2 atoms
1 bonds

Coords

1 0.5 0 0
2 -0.5 0 0

(...)
\end{lcverbatim}

\noindent A bond section also needs to be added, see this
\href{https://docs.lammps.org/molecule.html}{page} for details on the formatting of a
molecule template.
\end{tcolorbox}

